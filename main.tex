\documentclass[journal]{IEEEtran}

\usepackage{adjustbox}
\usepackage{algorithm}
\usepackage{algpseudocode}
\usepackage{amsfonts}
\usepackage{amsmath}
\usepackage{amssymb}
\usepackage{amsthm}
\usepackage{array}
\usepackage{circuitikz}
\usepackage{cite}
\usepackage{colortbl}
\usepackage{environ}
\usepackage{grffile}
\usepackage{hyperref}
\usepackage{import}
\usepackage{mathtools}
\usepackage{microtype}
\usepackage{multirow}
\usepackage{pgffor}
\usepackage{pgfplots}
\usepackage{siunitx}
\usepackage{stfloats}
\usepackage{tikz}
\usepackage{url}
\usepackage{xcolor}
\usepackage[T1]{fontenc}
\usepackage[caption=false,font=footnotesize,subrefformat=parens,labelformat=parens]{subfig}
\usepackage[short]{optidef}
% \usepackage[subtle]{savetrees}


\interdisplaylinepenalty=2500
\pgfplotsset{compat=newest}
\usepgfplotslibrary{patchplots}
\newtheorem{proposition}{Proposition}
\newtheorem{remark}{Remark}
\DeclareSIUnit{\belm}{Bm}
\DeclareSIUnit{\dBm}{\deci\belm}
\DeclareSIUnit{\beli}{Bi}
\DeclareSIUnit{\dBi}{\deci\beli}
\ctikzset{american}
\usetikzlibrary{arrows,calc,matrix,patterns,positioning}

\makeatletter
\tikzset{
	block/.style={draw,rectangle,align=center},
    from/.style args={#1 to #2}{
        above right={0cm of #1},
        /utils/exec=\pgfpointdiff
            {\tikz@scan@one@point\pgfutil@firstofone(#1)\relax}
            {\tikz@scan@one@point\pgfutil@firstofone(#2)\relax},
        minimum width/.expanded=\the\pgf@x,
        minimum height/.expanded=\the\pgf@y
	}
}
\makeatother

\algrenewcommand{\algorithmicwhile}{\textbf{While}}
\algrenewcommand{\algorithmicif}{\textbf{If}}
\algrenewcommand{\algorithmicthen}{\textbf{Then}}
\algrenewcommand{\algorithmicelse}{\textbf{Else}}
\algrenewcommand{\algorithmicend}{\textbf{End}}
\algrenewcommand{\algorithmicrepeat}{\textbf{Repeat}}
\algrenewcommand{\algorithmicuntil}{\textbf{Until}}


\begin{document}
	\title{Backscatter Modulation Design for Symbiotic Radio Networks}
	\maketitle

	\begin{section}{Backscatter Model}
		\begin{subsection}{Backscatter Principles}
			\begin{figure}[!t]
				\centering
				\subfloat[Block diagram]{
					\resizebox{0.6\columnwidth}{!}{
						\begin{circuitikz}[transform shape]
							\coordinate (O) at (0,0);
							\node[block,from={O to $(O) + (6.375,3.25)$}](T){};
							\draw (0,1.75)
								to[short] ++(-1,0)
								to[short] ++(0,0.75) node[bareantenna](A){Ant};
							\draw (0,1.75)
								to[short,-*] ++(0.25,0) coordinate(J);
							\draw (J)
								to[short] ++(0,1)
								to[short] ++(0.5,0) coordinate(J1);
							\node[block,from={$(J1) + (0,-0.25)$ to $(J1) + (2.5,0.25)$}](R){Rectifier};
							\draw (J)
								to[short] ++(0.5,0) coordinate(J2);
							\node[block,from={$(J2) + (0,-0.25)$ to $(J2) + (2.5,0.25)$}](D){Demodulator};
							\draw (J)
								to[short] ++(0,-1)
								to[short] ++(0.5,0) coordinate(J3)
								to [vR,mirror,invert,/tikz/circuitikz/bipoles/length=1cm] ++(1.75,0) node[ground,rotate=90]{};
							\node[block,from={$(J3) + (0,-0.25)$ to $(J3) + (2.5,0.25)$},draw=none](M){};
							\draw ($(J3) + (0,-0.25)$) rectangle ($(J3) + (2.5,0.25)$);
							\draw ($(J3) + (1.25,-0.5)$) node[]{Modulator};
							\draw[-{Latex[length=2mm]}] (R.east) -- ++(0.75,0);
							\draw[-{Latex[length=2mm]}] (D.east) -- ++(0.75,0);
							\draw[{Latex[length=2mm]}-] (M.east) -- ++(0.75,0);
							\node[block,from={$(R.east) + (0.75,-0.25)$ to $(R.east) + (2.875,0.25)$}](P){Power buffer};
							\node[block,from={$(M.east) + (0.75,-0.25)$ to $(D.east) + (2.875,0.25)$}](S){Digital\\section};
							\draw[dashed,-{Latex[length=2mm]}] (P.south) to (S.north);
							\coordinate (F1) at ($(P.south)!0.5!(S.north)$);
							\coordinate (F2) at ($(D.east)!0.5!(S.west)$);
							\coordinate (F3) at ($(D.south)!0.5!(M.north)$);
							\draw[dashed] (F1) to (F1-|D.north);
							\draw[dashed,-{Latex[length=2mm]}] (F1-|D.north) to (D.north);
							\draw[dashed] (F1-|F2) to (F2|-F3) to (M|-F3);
							\draw[dashed,-{Latex[length=2mm]}] (M|-F3) to (M.north);
						\end{circuitikz}
					}
					\label{fi:block_diagram}
				}
				\subfloat[One-port network model]{
					\resizebox{0.35\columnwidth}{!}{
						\begin{circuitikz}[transform shape]
							\draw (0,0) node[bareantenna](bareantenna){};
							\draw (bareantenna.west) ++(-1.5,0) node[waves](WI){};
							\draw (WI.north east) ++(0.25,0) node[font=\Large]{$\vec{E}_{\mathrm{I}}$};
							\draw (WI.south east) ++(0.25,0) node[font=\Large]{$\vec{H}_{\mathrm{I}}$};
							\draw (bareantenna.east) ++(0.875,0) node[waves](WR){};
							\draw (WR.north east) ++(0.25,0) node[font=\Large]{$\vec{E}_i$};
							\draw (WR.south east) ++(0.25,0) node[font=\Large]{$\vec{H}_i$};
							\draw (bareantenna)
								to [R=$Z_{\mathrm{A}}$,font=\Large,/tikz/circuitikz/bipoles/length=1cm] ++(0,-2)
								to [R=$Z_i$,font=\Large,/tikz/circuitikz/bipoles/length=1cm] ++(2,0) node[ground]{};
						\end{circuitikz}
					}
					\label{fi:one_port_network_model}
				}
				\caption{For a passive tag, the rectifier and demodulator rely on the incident electromagnetic wave for energy harvesting and downlink information decoding, while the load-switcher manipulate the reradiated signal for backscatter modulation.}
				\label{fi:tag}
			\end{figure}
			Consider a bistatic backscatter system that consists of an excitation source, a dedicated reader, and a passive tag. The excitation source generates a carrier wave signal, the dedicated reader decodes the tag message, and the tag simultaneously harvests energy, backscatters its own message, and demodulates the downlink information if necessary. As shown in Fig.~\subref*{fi:block_diagram}, a typical passive tag consists of a scattering antenna, an energy harvester, a integrated receiver\footnote{For example, \cite{Kim2021a} prototyped a compact-size pulse position demodulator based on an envelope detector, which brings great potential to coordination, synchronization, and reflection pattern control.}, a load-switching modulator, and on-chip components (e.g., micro-controller, memory, and sensors). A portion of the impinging signal is absorbed by the tag while the remaining is backscattered to the space, as illustrated in Fig.~\subref*{fi:one_port_network_model}. According to Green's decomposition \cite{Hansen1989}, the backscattered signal can be decomposed into the \emph{structural mode} component and the \emph{antenna mode} component. The former is fixed and depends on the antenna geometry and material properties\footnote{We assume the structural mode reflection can be modeled as part of the environment multipath and covered by channel estimation \cite{Boyer2014}.}, while the latter is adjustable and depends on the mismatch of the antenna and load impedance. Hence, the equivalent reflection coefficient at tag state $i$ is defined as\footnote{We assume the linear backscatter model where $\Gamma_i$ is irrelevant to the incident electromagnetic field at the tag \cite{Dobkin2012}.}
			\begin{equation}
				\Gamma_i = \frac{Z_i - Z_{\mathrm{A}}^*}{Z_i + Z_{\mathrm{A}}},
				\label{eq:reflection_coefficient}
			\end{equation}
			where $Z_i$ is the load impedance at state $i$ and $Z_{\mathrm{A}}$ is the antenna input impedance.
			\begin{remark}
				The reflection coefficient plays an important role in various network designs. For example, $\Gamma_i = 0$ (perfect matching) achieves maximum power transfer that is optimal for Wireless Power Transfer (WPT), $\lvert \Gamma_i \rvert = 1$ (perfect mismatching) achieves fully signal reflection that is optimal for Intelligent Reflecting Surface (IRS), and $\Gamma_i \ne \Gamma_j$ (adjustable matching) enables backscatter modulation.
			\end{remark}
		\end{subsection}

		\begin{subsection}{Backscatter Modulation}
			Tags perform backscatter modulation by switching the load impedance between different states. For $M$-ary Phase Shift Keying (PSK), the reflection coefficient $\Gamma_i$ maps to the desired signal constellation point $c_i$ as \cite{Thomas2012a}
			\begin{equation}
				\Gamma_i = \alpha c_i = \alpha e^{\jmath \left(\frac{2 \pi i}{M} + \phi\right)},
				\label{eq:backscatter_modulation}
			\end{equation}
			where $\alpha \in [0,1]$ is the reflection efficiency at a given direction, and $\phi$ is a fixed phase offset.
			\begin{remark}
				For passive tags, the reflection efficiency $\alpha$ controls the tradeoff between the backscatter strength and harvestable power. Interestingly, when $\alpha = 1$, the reflection coefficient set $\{\Gamma_i\}$ of the $M$-PSK backscatter coincides with that of an ideal discrete $M$-state uniform IRS. The optimal strategy for the IRS is to choose one reflection state with probability \num{1} to boost the equivalent channel, while the optimal strategy for the modulator is to utilize all constellation points with equal probability. It inspires one to adaptively design the p.m.f. of tag symbols to jointly benefit the backscatter modulation and passive beamforming.
			\end{remark}
		\end{subsection}
	\end{section}

	\begin{section}{System Model}
		\begin{figure}[!t]
			\centering
			\def\svgwidth{0.9\columnwidth}
			\import{assets/}{symbiotic_radio.pdf_tex}
			\caption{A single-user single-tag symbiotic radio system.}
			\label{fi:symbiotic_radio}
		\end{figure}
		As shown in Fig.~\ref{fi:symbiotic_radio}, we propose a single-user (UE) single-tag (TG) symbiotic radio network where the RF signal generated by the single-antenna Access Point (AP) is shared by two coexisting systems. In the primary AP-UE downlink system, the AP transmits to the single-antenna user. In the secondary AP-TG-UE backscatter system, the AP acts as the carrier emitter, the user serves as the backscatter reader, and the single-antenna tag modulates its information over the reradiated RF signal by varying the reflection coefficient. Denote the AP-UE direct channel as $h_{\mathrm{D}}$, the AP-TG forward channel as $h_{\mathrm{F}}$, and the TG-UE backward channel as $h_{\mathrm{B}}$. We consider the quasi-static block fading model and assume the CSI of the direct channel and the cascaded forward-backward channel $h_{\mathrm{C}} \triangleq h_{\mathrm{B}} h_{\mathrm{F}}$ are known at the AP.\footnote{Due to the lack of RF chains at the passive tag, accurate and efficient CSI acquisition at the AP can be challenging. One possible approach is that the AP sends known pilots, the tag responds in a pre-defined manner, and the user performs least-square estimation then feeds back to the AP \cite{Bharadia2015,Yang2015b,Guo2019e}.} It is assumed that the primary symbol $s$ follows standard CSCG distribution $\mathcal{CN}(0,1)$ and the secondary symbol $c$ employs $M$-PSK modulation by \eqref{eq:backscatter_modulation}. To provide a preliminary insight, we consider a parasitic symbiotic radio \cite{Long2020a} where the primary and secondary symbol periods are equal. \footnote{However, parasitic symbiotic radio requires fast tag-state switching and frequent synchronization, which can be challenging for passive tags.} The user simultaneously captures the signal from both primary and secondary links as\footnote{We assume the time difference of arrival from the AP-UE path and the AP-TG-UE path are negligible compared to the symbol period \cite{Guo2019b,Liang2020,Long2020a}.}
		\begin{equation}
			y = \sqrt{P} h_{\mathrm{D}} s + \sqrt{\alpha P} h_{\mathrm{C}} s c + n,
		\end{equation}
		where $P$ is the average transmit power of the AP and $n \sim \mathcal{CN}(0,1)$ is the additive white Gaussian noise.

		\begin{remark}
			The symbiotic radio network can be regarded as as a special case of Multiple Access Channel (MAC) because the AP and the tag simultaneously transmit to the user. It is known that Superposition Coding-Successive Interference Cancellation (SC-SIC) with different decoding orders can achieve different vertices of the MAC capacity region \cite{Goldsmith2005}. Therefore, most relevant papers proposed the user to decode the primary message first (by treating the tag interference as noise), cancel out its contribution from the received signal, then decode the secondary message. Since the direct channel is typically much stronger than the cascaded channel \cite{Ozdogan2020}, the primary decoding is expected to enjoy a high Signal-to-Interference-and-Noise Ratio (SINR) and the secondary decoding is ideally interference-free.
		\end{remark}

		To investigate how backscatter modulation potentially benefits the primary transmission, we instead start decoding from the secondary link. Once the tag message is successfully recovered, by combining the backscattered symbol with the cascaded channel, the uncertainty of the AP-TG-UE path can be eliminated, and its contribution to primary transmission can be modeled within the equivalent AP-UE channel. This is reminiscent of IRS-aided point-to-point transmission. Overall, the passive tag not only embeds its own message in the reflection pattern, but also influences the legacy transmission.

		\begin{remark}
			The main difference between symbiotic radio and conventional MAC is that the primary message also reaches the user from the backscatter link. This characteristic inspires one to decode the tag message first and model its contribution within the equivalent channel (i.e., unify backscatter decoding and channel training), instead of performing SIC. In such case, the primary transmission is able to achieve ergodic capacity with artificial channel variation created by the backscatter modulation.
		\end{remark}

		% channel training - backscatter decoding

		% \begin{remark}
		% 	To investigate how backscatter modulation potentially benefits the primary transmission, we instead decode the secondary link before the primary link. Once the tag message is successfully decoded, the uncertainty of the AP-TG-UE path can be perfectly removed (by combining the backscattered symbol and the cascaded channel), and its contribution can be modeled within the equivalent AP-UE channel. This is reminiscent of IRS-aided point-to-point transmission with CSI uncertainty (by backscatter modulation instead of channel estimation), where the passive tag simultaneously embeds its own message (embodied in secondary decoding) and assists the legacy transmission (embodied in primary decoding).

		% 	[Channel over channel, randomness over randomness]
		% \end{remark}

		% Define $s_k$ as the primary symbol for user $k$ with $\boldsymbol{s} \triangleq [s_1,\ldots,s_K]^H \in \mathbb{C}^{K \times 1}$ satisfying $\mathbb{E}\{\boldsymbol{s}\boldsymbol{s}^H\}=\boldsymbol{I}$, and $c_{l_k}$ as the secondary symbol of tag $l_k$ with $\boldsymbol{c}_k \triangleq [c_1,\ldots,c_{L_k}]^H \in \mathbb{C}^{L_k \times 1}$ satisfying $\mathbb{E}\{\boldsymbol{c}_k\boldsymbol{c}_k^H\}=\boldsymbol{I}$. Let $s_k \sim \mathcal{CN} (0,1)$ and $c_{l_k}$ employs $M$-PSK by \eqref{eq:reflection_coefficient}. As tags typically transmit at a much lower rate than the AP, we assume the secondary symbol period $T_c$ is $N \in \mathbb{Z}_{++}$ times the primary symbol period $T_s$ (i.e., $T_c=NT_s$) and focus on one particular secondary symbol period. For linear precoding with no dedicated tag beams, the transmit signal during the primary symbol period $n \in \mathcal{N} \triangleq \{1,\ldots,N\}$ is
		% \begin{equation}
		% 	\boldsymbol{x}(n) = \sum_{k=1}^K \boldsymbol{w}_k s_k(n),
		% \end{equation}
		% where $\boldsymbol{w}_k$ is the precoder for user $k$ satisfying the average transmit power constraint $\sum_{k=1}^K \lVert \boldsymbol{w}_k \rVert^2 \le P$. The received signal at tag $l_k$ is\footnote{For simplicity, the backscattered signals from all other tags are modeled within noise.}
		% \begin{equation}
		% 	y_{l_k}(n) = \sum_{j=1}^K \boldsymbol{h}_{\mathrm{F},l_k}^H \boldsymbol{w}_j s_j(n) + z_{l_k}(n),
		% \end{equation}
		% where $z_{l_k} \sim \mathcal{CN} (0,\sigma_{l_k}^2)$ is the additive white Gaussian noise\footnote{Since the tag employs no active RF components, the noise captured at backscatter antenna $z_{l_k}$ is assumed negligible \cite{Nikitin2005,Arnitz2013,Wu2021b}.}. Hence, the corresponding antenna input power is
		% \begin{equation}
		% 	P_{0,l_k} = \sum_{j=1}^K \lvert \boldsymbol{h}_{\mathrm{F},l_k}^H \boldsymbol{w}_j \rvert^2.
		% 	\label{eq:P_0}
		% \end{equation}
		% In practice, the rectifier is a non-linear device and the RF-to-DC conversion efficiency depends on the input power \cite{Clerckx2016a,Clerckx2018,Clerckx2019}. On top of \eqref{eq:P_H} and \eqref{eq:P_0}, when all modulation states are equiprobable, the average harvester output power is
		% \begin{align}
		% 	P_{l_k}
		% 	& = \frac{1}{M} \Biggl( \sum_{i_1} \zeta \biggl( \eta_{i_1} (1 - \alpha_{l_k}^2) \sum_{j=1}^K \lvert \boldsymbol{h}_{\mathrm{F},l_k}^H \boldsymbol{w}_j \rvert^2 \biggr) \\
		% 	& \quad + \sum_{i_2} \zeta \biggl( \frac{(1 - \alpha_{l_k}^2) \sum_{j=1}^K \lvert \boldsymbol{h}_{\mathrm{F},l_k}^H \boldsymbol{w}_j \rvert^2}{\eta_{i_2}} \biggr) \Biggr),
		% \end{align}
		% where $i_1 \ge M {\arccos \alpha_{l_k}} / {2 \pi}$, $i_2<M {\arccos \alpha_{l_k}} / {2 \pi}$, and $\zeta(\cdot)$ is the harvester response function. On the other hand, we define the scattered channel by tag $l_k$ as
		% \begin{equation}
		% 	\boldsymbol{h}_{l_k}^H \triangleq \alpha_{l_k} h_{\mathrm{B},l_k} \boldsymbol{h}_{\mathrm{F},l_k}^H.
		% \end{equation}
		% Hence, the received signal at user $k$ is
		% \begin{align}
		% 	y_k(n)
		% 	& = \sum_{j=1}^K \left( \boldsymbol{h}_{\mathrm{D},k}^H + \sum_{l_k=1}^{L_k} c_{l_k} \boldsymbol{h}_{l_k}^H \right) \boldsymbol{w}_j s_j(n) + z_k(n) \\
		% 	& \triangleq \sum_{j=1}^K \boldsymbol{h}_k^H \boldsymbol{w}_j s_j(n) + z_k(n),
		% \end{align}
		% where
		% \begin{equation}
		% 	\boldsymbol{h}_k^H \triangleq \boldsymbol{h}_{\mathrm{D},k}^H + \sum_{l_k=1}^{L_k} c_{l_k} \boldsymbol{h}_{l_k}^H
		% 	\label{eq:h}
		% \end{equation}
		% is the equivalent channel of user $k$ and $z_k \sim \mathcal{CN} (0,\sigma_k^2)$ is the additive white Gaussian noise.
		% \begin{remark}
		% 	From the perspective of broadcast, the components scattered by the tags essentially create uncertainty over the direct AP-UE channel. If the backscattered symbols are unknown when decoding the broadcasted symbols, the uncertainty should be either covered by non-coherent detection \cite{Long2020a} or modeled as CSI error that introduces interference. That is to say, the equivalent channel is perfectly known only when all tag massages are successfully decoded.
		% \end{remark}

		% \begin{figure}[!t]
		% 	\centering
		% 		\resizebox{0.95\columnwidth}{!}{
		% 			\begin{tikzpicture}
		% 				\foreach \x in {1,...,12}
		% 					\draw [pattern=crosshatch dots,pattern color=blue](\x,0) rectangle ++(1,1);
		% 				\foreach \x in {1,...,12}
		% 					\foreach \y in {-3,...,-1}{
		% 						\pgfmathsetmacro\floorval{int(floor((\x+\y)/4))}
		% 						\pgfmathsetmacro\modval{int(mod((\x+\y),4))}
		% 						\ifnum \floorval=0
		% 							\ifnum \modval=0
		% 								\draw [pattern=horizontal lines,pattern color=red](\x,\y) rectangle ++(1,1);
		% 							\else
		% 								\draw [pattern=horizontal lines,pattern color=yellow](\x,\y) rectangle ++(1,1);
		% 							\fi
		% 						\else
		% 							\ifnum \floorval=1
		% 								\ifnum \modval=0
		% 									\draw [pattern=north west lines,pattern color=red](\x,\y) rectangle ++(1,1);
		% 								\else
		% 									\draw [pattern=north west lines,pattern color=yellow](\x,\y) rectangle ++(1,1);
		% 								\fi
		% 							\else
		% 								\ifnum \floorval=2
		% 									\ifnum \modval=0
		% 										\draw [pattern=vertical lines,pattern color=red](\x,\y) rectangle ++(1,1);
		% 									\else
		% 										\draw [pattern=vertical lines,pattern color=yellow](\x,\y) rectangle ++(1,1);
		% 									\fi
		% 								\else
		% 									\draw (\x,\y) rectangle ++(1,1);
		% 									% \ifnum \modval=0
		% 									% 	\draw (\x,\y) rectangle ++(1,1);
		% 									% \else
		% 									% 	\draw [pattern=north east lines,pattern color=yellow](\x,\y) rectangle ++(1,1);
		% 									% \fi
		% 								\fi
		% 							\fi
		% 						\fi
		% 					}
		% 				\node[font=\huge] at (-2,0.5) {Broadcast};
		% 				\node[font=\huge] at (-2,-0.5) {Tag \num{1}};
		% 				\node[font=\huge] at (-2,-1.5) {Tag \num{2}};
		% 				\node[font=\huge] at (-2,-2.5) {Tag \num{3}};
		% 				\node[font=\huge] at (14,-1) {$\cdots$};
		% 				\node[font=\huge,right] at (16.75,0){$b=1$};
		% 				\node[font=\huge,right] at (16.75,-1){$b=2$};
		% 				\node[font=\huge,right] at (16.75,-2){$b=3$};
		% 				\draw [pattern=horizontal lines](16,-0.25) rectangle ++(0.5,0.5);
		% 				\draw [pattern=north west lines](16,-1.25) rectangle ++(0.5,0.5);
		% 				\draw [pattern=vertical lines](16,-2.25) rectangle ++(0.5,0.5);
		% 				\draw[<->] (1,1.5) -- (2,1.5) node[font=\huge,midway,above]{$T_s$};
		% 				\draw[<->] (1,-3.5) -- (5,-3.5) node[font=\huge,midway,below=0.125]{$T_c=NT_s$};
		% 				\draw[->] (10.5,1.625) -- (10.5,1.25) node[font=\huge,at start,above]{Block $r$};
		% 			\end{tikzpicture}
		% 		}
		% 	\caption{Primary and secondary symbol blocks received by user $k$ for $N=4$ and $L_k=3$. The blocks are normalized to the primary symbol period $T_s$. In the broadcast link, each block contains the message of all users at a specific time. In the backscatter links, each tag symbol lasts $N$ blocks (illustrated by a specific pattern) and the tag streams are delayed by multiples of $T_s$. Apart from decoding the self message, each user also decodes at most one tag message per primary symbol period (denoted by red blocks) and regards it as part of known channel in the following $N-1$ blocks.}
		% 	\label{fi:blocks}
		% \end{figure}

		% To mitigate the interference from the backscatter link, we utilize its low rate property and propose a joint user and tag multiple access scheme as illustrated in Fig.~\ref{fi:blocks}. Specifically, the broadcast system can employ any general multiple access scheme\footnote{Since tags are associated to the nearest user, non-orthogonal protocols (e.g., NOMA and MU-LP) involving less coordination among users and tags are preferred \cite{Ding2021}.} (denoted by blue blocks), while the backscatter system integrates TDMA and NOMA as follows. Assuming the number of tags associated to user $k$ satisfies $L_k \le N$, $\forall k \in \mathcal{K}$\footnote{The proposed scheme can be extended to case of $L_k > N$, where at most $\lceil L_k/N \rceil$ tags transmit synchronously and user $k$ decodes up to $\lceil L_k/N \rceil$ tags per primary symbol period. Here we assume $L_k \le N$ for simplicity.}, we intentionally delay the stream of tag $l_k$ by $l_k-1$ primary symbol periods such that all tags transmit asynchronously and non-orthogonally. In such case, each user only needs to decode at most one tag message per primary symbol period (denoted by red blocks) and models it as part of the estimated channel in the following $N-1$ primary symbol periods (denoted by yellow blocks). We denote $r \in \mathbb{Z}_{++}$ as the index of primary symbols, and define $b \triangleq \lceil r/N \rceil$ as the index of secondary symbols and $d \equiv r \pmod N$ as the remainder of $N$ divides $r$. Therefore, the backscattered symbol by tag $l_k$ at block $r$ can be related to its index $b$ as
		% \begin{equation}
		% 	c_{l_k}(r) =
		% 	\begin{cases}
		% 		c_{l_k}(b), & d = 0 \text{ or } d \ge l_k, \\
		% 		c_{l_k}(b-1), & 0 < d < l_k.
		% 	\end{cases}
		% \end{equation}
		% It is assumed that $c_{l_k}(0)$, $\forall l_k \in \mathcal{L}_k$ is known at user $k$ and can be designed as zeros or pilots or cyclic prefixes.

		% At block $r$, user $k$ always decodes its $r$-th primary symbol $s_k(r)$ first. If $d=0$ or $d>L_k$ \footnote{Note that $d=0$ corresponds to $n=N$.}, all backscattered symbols would be the same as in the previous block and no uncertainty is introduced by the backscattered link (denoted by all-yellow columns). Therefore, the equivalent channel at block $r$ is perfectly known and the tags create no interference to the primary transmission. On the other hand, if $0 < d \le L_k$, tag $d$ would trigger a new symbol $c_d(b)$ that lasts $N$ blocks but should be decoded within the current block $r$. In this case, user $k$ first treats the channel uncertainty from $c_d(b)$ as interference during primary decoding, then cancels the interference from decoded primary symbols and detects $c_d(b)$. Once a backscattered symbol is successfully decoded, its contribution to the equivalent channel \eqref{eq:h} is known. Therefore, under perfect detection, the estimated channel of user $k$ at block $r$ is
		% \begin{equation}
		% 	\bar{\boldsymbol{h}}_k^H(r) =
		% 		\begin{cases}
		% 			\boldsymbol{h}_{\mathrm{D},k}^H + \boldsymbol{c}_k^H(r) \boldsymbol{H}_k, & d = 0 \text{ or } d > L_k, \\
		% 			\boldsymbol{h}_{\mathrm{D},k}^H + \boldsymbol{c}_{k \setminus d}^H(r) \boldsymbol{H}_{k \setminus d}, & 0 < d \le L_k,
		% 		\end{cases}
		% 		\label{eq:h_bar}
		% \end{equation}
		% where we define $\boldsymbol{c}_{k \setminus d} \triangleq [c_1,\ldots,c_{d-1},c_{d+1},\ldots,c_{L_k}]^H \in \mathbb{C}^{(L_k-1) \times 1}$, $\boldsymbol{H}_k \triangleq [\boldsymbol{h}_1,\ldots,\boldsymbol{h}_{L_k}]^H \in \mathbb{C}^{L_k \times Q}$ and $\boldsymbol{H}_{k \setminus d} \triangleq [\boldsymbol{h}_1,\ldots,\boldsymbol{h}_{d-1},\boldsymbol{h}_{d+1},\ldots,\boldsymbol{h}_{L_k}]^H \in \mathbb{C}^{(L_k-1) \times Q}$. Besides, the corresponding channel uncertainty is
		% \begin{equation}
		% 	\tilde{\boldsymbol{h}}_k^H(r) = \boldsymbol{h}_k^H(r) - \bar{\boldsymbol{h}}_k^H(r) =
		% 		\begin{cases}
		% 			0, & d = 0 \text{ or } d > L_k, \\
		% 			c_{d}(r) \boldsymbol{h}_{d}^H, & 0 < d \le L_k.
		% 		\end{cases}
		% 		\label{eq:h_tilde}
		% \end{equation}
		% \begin{remark}
		% 	From the perspective of broadcast, the backscattered channel is similar to the auxiliary channel provided by IRS. Once a tag symbol is successfully decoded, the corresponding channel uncertainty can be mitigated in the following $N-1$ primary symbol periods. This decoding mechanism can be viewed as a backscatter channel training process where the cascaded channels remain unchanged and the tag reflection coefficients are learned. If $N \gg L_k$, the training overhead (corresponding to $0 < d \le L_k$) can be omitted where the tags backscatter at very low rates and behave similar to the IRS elements. That is to say, passive beamforming can be enabled by carefully choosing the backscatter symbols based on CSI.
		% \end{remark}

		% Next, we investigate the performance of the proposed protocol under typical primary multiple access schemes.

		% \begin{subsection}{NOMA}
		% 	We first consider NOMA for the primary link. Without loss of generality, we assume the SNR of the estimated channels at block $r$ satisfy $\lVert \bar{\boldsymbol{h}}_j(r) \rVert / \sigma_j \le \lVert \bar{\boldsymbol{h}}_k(r) \rVert / \sigma_k$ such that the downlink symbol $s_j(r)$ is decoded before $s_k(r)$, $\forall j < k, \, j,k \in \mathcal{K}$. The SINR for user $k$ to decode the message of user $j$ at block $r$ is
		% 	\begin{equation}
		% 		\gamma_{j \to k}^{\mathrm{(N)}}(r) = \frac{\lvert \bar{\boldsymbol{h}}_k^H(r) \boldsymbol{w}_j \rvert^2}{\sum_{i=j+1}^K \lvert \bar{\boldsymbol{h}}_k^H(r) \boldsymbol{w}_i \rvert^2 + \sum_{i=1}^K \lvert \tilde{\boldsymbol{h}}_k^H(r) \boldsymbol{w}_i \rvert^2 + \sigma_k^2},
		% 		\label{eq:gamma_jk^N}
		% 	\end{equation}
		% 	where the interference from all users stronger than $j$ and channel uncertainty from all tags are treated as noise. Once $s_j(r)$ is successfully decoded, its contribution is cancelled from $y_k(r)$. The process is repeated until user $k$ reaches its own stream, and the self-decoding SINR is
		% 	\begin{equation}
		% 		\gamma_{k \to k}^{\mathrm{(N)}}(r) = \frac{\lvert \bar{\boldsymbol{h}}_k^H(r) \boldsymbol{w}_k \rvert^2}{\sum_{j=k+1}^K \lvert \bar{\boldsymbol{h}}_k^H(r) \boldsymbol{w}_j \rvert^2 + \sum_{j=1}^K \lvert \tilde{\boldsymbol{h}}_k^H(r) \boldsymbol{w}_j \rvert^2 + \sigma_k^2}.
		% 		\label{eq:gamma_kk^N}
		% 	\end{equation}
		% 	After successfully decoded its own stream, user $k$ has two options. When $d=0$ or $d>L_k$, the estimated channel \eqref{eq:h_bar} equals the equivalent channel \eqref{eq:h} and it boils down to an backscatter-assisted NOMA broadcast block. In this case, the second term of the denominator of \eqref{eq:gamma_jk^N}, \eqref{eq:gamma_kk^N} are zero and user $k$ terminates the operation after decoding its own stream. On the other hand, when $0 < d \le L_k$, tag $d$ triggers a new symbol at block $r$ that contributes to channel uncertainty \eqref{eq:h_tilde}. In this case, user $k$ continues to cancel the interference from its own stream (i.e., perform an extra layer SIC on top of primary decoding), then decodes $c_d(b)$ at SINR
		% 	\begin{equation}
		% 		\gamma_{d \to k}^{\mathrm{(N)}}(r) = \frac{\sum_{j=1}^K \lvert \boldsymbol{h}_d^H \boldsymbol{w}_j \rvert^2}{\sum_{j=k+1}^K \lvert \bar{\boldsymbol{h}}_k^H(r) \boldsymbol{w}_j \rvert^2 + \sigma_k^2}.
		% 		\label{eq:gamma_dk^N}
		% 	\end{equation}
		% 	Once $c_d(b)$ is successfully decoded, its contribution would be modeled within the known channel in the following $N-1$ blocks to help both primary and secondary decodings.
		% \end{subsection}

		% \begin{subsection}{MU-LP}
		% 	MU-LP in the primary link treats the interference occurred in primary decoding as interference. The SINR for user $k$ to decode its own stream at block $r$ is
		% 	\begin{equation}
		% 		\gamma_{k \to k}^{\mathrm{(M)}}(r) = \frac{\lvert \bar{\boldsymbol{h}}_k^H(r) \boldsymbol{w}_k \rvert^2}{\sum_{j \in \mathcal{K} \setminus k} \lvert \bar{\boldsymbol{h}}_k^H(r) \boldsymbol{w}_j \rvert^2 + \sum_{j=1}^K \lvert \tilde{\boldsymbol{h}}_k^H(r) \boldsymbol{w}_j \rvert^2 + \sigma_k^2}.
		% 		\label{eq:gamma_kk^M}
		% 	\end{equation}
		% 	Similarly, when $0 < d \le L_k$, user $k$ cancels the interference from its own stream and decodes $c_d(b)$ at SINR
		% 	\begin{equation}
		% 		\gamma_{d \to k}^{\mathrm{(N)}}(r) = \frac{\sum_{j=1}^K \lvert \boldsymbol{h}_d^H \boldsymbol{w}_j \rvert^2}{\sum_{j \in \mathcal{K} \setminus k} \lvert \bar{\boldsymbol{h}}_k^H(r) \boldsymbol{w}_j \rvert^2 + \sigma_k^2}.
		% 		\label{eq:gamma_dk^M}
		% 	\end{equation}
		% 	The contribution of $c_d(b)$ is then modeled within the estimated channel in the following $N-1$ blocks.
		% \end{subsection}

		% \begin{remark}
		% 	For a specific channel realization, we notice that $\gamma_{j \to k}$ and $\gamma_{k \to k}$ are generally functions of $\boldsymbol{c}_k$ such that the primary rate at each block should be evaluated over the expectation of the backscattered symbols \cite{Long2020a,Zhang2020c}. On the other hand, how tags contribute to the estimated channel depends on the specific block. Since each decoding cycle lasts $N$ blocks, another expectation over $N$ blocks is necessary. For a sufficiently large $N$, the average primary achievable rate of user $k$ is
		% 	\begin{equation}
		% 		\mathbb{E}_n \Bigl\{ \mathbb{E}_{\boldsymbol{c}_k}\bigl\{R_{k \to k}(n)\bigr\} \Bigr\} = \frac{1}{N} \sum_{n=1}^N \mathbb{E}_{\boldsymbol{c}_k}\Bigl\{\log_2 \bigl( 1 + \gamma_{k \to k}(n) \bigr)\Bigr\}.
		% 		% \mathbb{E}_n \bigl\{ \mathbb{E}_{\boldsymbol{c}_k}\{\gamma_{k \to k}\} \bigl\} = \frac{1}{N} \sum_{n=1}^N \mathbb{E}_{\boldsymbol{c}_k}\{\gamma_{k \to k}(n)\}.
		% 	\end{equation}
		% 	In contrast, although $\gamma_{l_k \to k}$ depends on $\boldsymbol{c}_{k \setminus l_k}$, the decoding process for tag $l_k$ only lasts one block such that $\gamma_{l_k \to k}$ is irrelevant to $N$. Hence, the average secondary SINR of tag $l_k$ is $\mathbb{E}_{\boldsymbol{c}_{k \setminus l_k}}\{\gamma_{l_k \to k}\}$. [Need rate-SINR relationship for M-PSK.]
		% \end{remark}
		\label{se:system_model}
	\end{section}

	\bibliographystyle{IEEEtran}
	\bibliography{IEEEabrv,library.bib}
\end{document}
