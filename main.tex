\documentclass[journal]{IEEEtran}

\usepackage{adjustbox}
\usepackage{algorithm}
\usepackage{algpseudocode}
\usepackage{amsfonts}
\usepackage{amsmath}
\usepackage{amssymb}
\usepackage{amsthm}
\usepackage{array}
\usepackage{circuitikz}
\usepackage{cite}
\usepackage{colortbl}
\usepackage{environ}
\usepackage{grffile}
\usepackage{hyperref}
\usepackage{import}
\usepackage{mathtools}
\usepackage{microtype}
\usepackage{multirow}
\usepackage{pgffor}
\usepackage{pgfplots}
\usepackage{siunitx}
\usepackage{stfloats}
\usepackage{tikz}
\usepackage{url}
\usepackage{xcolor}
\usepackage[T1]{fontenc}
\usepackage[caption=false,font=footnotesize,subrefformat=parens,labelformat=parens]{subfig}
\usepackage[short]{optidef}
% \usepackage[subtle]{savetrees}


\interdisplaylinepenalty=2500
\pgfplotsset{compat=newest}
\usepgfplotslibrary{patchplots}
\newtheorem{proposition}{Proposition}
\newtheorem{remark}{Remark}
\DeclareSIUnit{\belm}{Bm}
\DeclareSIUnit{\dBm}{\deci\belm}
\DeclareSIUnit{\beli}{Bi}
\DeclareSIUnit{\dBi}{\deci\beli}
\ctikzset{american}
\usetikzlibrary{arrows,matrix,positioning,patterns}


\algrenewcommand{\algorithmicwhile}{\textbf{While}}
\algrenewcommand{\algorithmicif}{\textbf{If}}
\algrenewcommand{\algorithmicthen}{\textbf{Then}}
\algrenewcommand{\algorithmicelse}{\textbf{Else}}
\algrenewcommand{\algorithmicend}{\textbf{End}}
\algrenewcommand{\algorithmicrepeat}{\textbf{Repeat}}
\algrenewcommand{\algorithmicuntil}{\textbf{Until}}


\begin{document}
	\title{Backscatter Modulation Design for Symbiotic Radio Networks}
	\maketitle

	\begin{section}{Backscatter Model}
		\begin{subsection}{Backscatter Principles}
			\begin{figure}[!t]
				\centering
				\subfloat[Backscatter antenna]{
					\resizebox{0.3\columnwidth}{!}{
						\begin{circuitikz}[transform shape]
							\draw (0,0) node[bareantenna](bareantenna){};
							\draw (bareantenna.west) ++(-1.5,0) node[waves](WI){};
							\draw (WI.north east) ++(0.25,0) node[]{$\vec{E}_{\mathrm{I}}$};
							\draw (WI.south east) ++(0.25,0) node[]{$\vec{H}_{\mathrm{I}}$};
							\draw (bareantenna.east) ++(0.875,0) node[waves](WR){};
							\draw (WR.north east) ++(0.25,0) node[]{$\vec{E}_i$};
							\draw (WR.south east) ++(0.25,0) node[]{$\vec{H}_i$};
							\draw (bareantenna)
								to [R=$Z_{\mathrm{A}}$] ++(0,-2)
								to [R=$Z_i$] ++(3,0) node[ground](GND){};
						\end{circuitikz}
					}
					\label{ci:backscatter_antenna}
				}
				\subfloat[Tag equivalent circuit]{
					\resizebox{0.65\columnwidth}{!}{
						\begin{circuitikz}[transform shape]
							\draw (0,0) coordinate(O)
								to [sV,l=$V_0$] ++(0,1.5)
								to [L,l=$X_{\mathrm{A}}$] ++(0,1.5)
								to [R=$R_{\mathrm{R}}$,-*] ++(3,0) coordinate(AM)
								to [sI,l_=$I_0$,-*] ++(0,-3)
								to [short] (O);
							\draw (AM)
								to [short] ++(0.5,0)
								to [L,l=$X_i^{\mathrm{M}}$,-*] ++(3,0) coordinate(M)
								to [R=$R_{i,2}^{\mathrm{M}}$] ++(3,0) coordinate(MH);
							\draw (M)
								to [R=$R_{i,1}^{\mathrm{M}}$,-*] ++(0,-3);
							\draw (MH)
								to [D,-*] ++(1.5,0) coordinate(H)
								to [C=$X_{\mathrm{H}}$,-*] ++(0,-3);
							\draw (H)
								to [short] ++(1.5,0)
								to [R=$R_{\mathrm{C}}$] ++(0,-3)
								to [short] (O);

							\draw [blue,dashed] (-1.25,-0.5) rectangle (3.5,3.75);
							\draw [yellow,dashed] (4,-0.5) rectangle (9,3.75);
							\draw [red,dashed] (9.5,-0.5) rectangle (13.5,3.75);

							\draw [blue] (1.125,4) node[]{Antenna};
							\draw [yellow] (6.5,4) node[]{Modulator};
							\draw [red] (11.5,4) node[]{Harvester + Chip};

							\draw (3.375,2.675) to [short] (3.375,3.175) to [short] (3.25,3.175) to [short,i_=$Z_{\mathrm{A}}$] (3.125,3.175);
							\draw (4.125,-0.125) to [short] (4.125,0.375) to [short] (4.25,0.375) to [short,i=$Z_i$] (4.375,0.375);
							\draw (9.625,-0.125) to [short] (9.625,0.375) to [short] (9.75,0.375) to [short,i=$Z_{\mathrm{H}}$] (9.875,0.375);
						\end{circuitikz}
					}
					\label{ci:tag_equivalent_circuit}
				}
				\caption{The backscatter antenna behaves as a constant power source. In the tag equivalent circuit, the blue, yellow and red block represent respectively the lossless minimum scattering antenna \cite{Huang2021}, the dual-load matching network as modulator \cite{Ebrahimi-Asl2018}, and the harvester feeding the chip \cite{Clerckx2016a}.}
				\label{fi:tag}
			\end{figure}
			Consider a bistatic backscatter system that consists of an excitation source, a backscatter reader, and a passive tag. The excitation source generates a carrier wave signal, the backscatter reader detects the tag message, and the tag employs the impinging electromagnetic wave for information modulation and energy harvesting. A typical passive tag consists of a backscatter antenna, a matching network, an energy harvester, and on-chip components (e.g., microcontroller, memory, sensor, and integrated receiver\footnote{For example, a compact-size integrated receiver was prototyped in \cite{Kim2021a} to perform pulse position demodulation by an envelope detector. It enables coordination, synchronization, and reflection pattern control.}). The backscatter antenna behaves as a constant power source, where the voltage is induced by the incident electric field $\vec{E}_{\mathrm{I}}$ and the current is induced by the incident magnetic field $\vec{H}_{\mathrm{I}}$ \cite{Paul2006}. As shown in Fig.~\subref*{ci:backscatter_antenna}, a portion of the incident power is absorbed by the tag while the remaining is backscattered to the space. The backscattered signal can be decomposed into the \emph{structural mode} component and the \emph{antenna mode} component \cite{Hansen1989}. The former is fixed and depends on the antenna geometry and material properties\footnote{We assume the structural mode reflection can be modeled as part of the environment multipath and covered by channel estimation \cite{Boyer2014}.}, while the latter is adjustable and depends on the mismatch of the antenna and load impedance. Hence, the equivalent reflection coefficient at tag state $i$ is defined as\footnote{We assume the linear backscatter where $\Gamma_i$ is irrelevant to the incident electromagnetic field at the tag \cite{Dobkin2012}.}
			\begin{equation}
				\Gamma_i = \frac{Z_i - Z_{\mathrm{A}}^*}{Z_i + Z_{\mathrm{A}}},
				\label{eq:reflection_coefficient}
			\end{equation}
			where $Z_i = R_i + \jmath X_i$ is the antenna load impedance at state $i$, $Z_{\mathrm{A}} = R_{\mathrm{R}} + R_{\mathrm{loss}} + \jmath X_{\mathrm{A}}$ is the antenna input impedance, $R_{\mathrm{R}}$ is the radiation resistance that measures the antenna energy efficiency, $R_{\mathrm{loss}}$ is the loss resistance that models the ohmic or dielectric loss, and $X_{\mathrm{A}}$ is the antenna reactance \cite{Huang2021}.
			\begin{remark}
				The reflection coefficient plays an important role in various network designs. For example, $\Gamma_i = 0$ (perfect matching) achieves maximum power transfer that is optimal for Wireless Power Transfer (WPT), $\lvert \Gamma_i \rvert = 1$ (perfect mismatching) achieves fully signal reflection that is optimal for Intelligent Reflecting Surface (IRS), and $\Gamma_i \ne \Gamma_j$ (adjustable matching) enables backscatter modulation.
			\end{remark}
		\end{subsection}

		\begin{subsection}{Backscatter Modulation}
			Tags perform backscatter modulation by switching the load impedance between different states. For $M$-ary Phase Shift Keying (PSK), the reflection coefficient $\Gamma_i$ maps to the desired signal constellation point $c_i$ as \cite{Thomas2012a}
			\begin{equation}
				\Gamma_i = \alpha c_i = \alpha e^{\jmath \left(\frac{2 \pi i}{M} + \phi\right)},
				\label{eq:backscatter_modulation}
			\end{equation}
			where $\alpha \in [0,1]$ is the reflection efficiency at a given direction, and $\phi$ is a fixed phase offset.
			\begin{remark}
				For passive tags, the reflection efficiency $\alpha$ controls the tradeoff between the backscatter strength and harvestable power. Interestingly, when $\alpha = 1$, the reflection coefficient set $\{\Gamma_i\}$ of the $M$-PSK backscatter coincides with that of an ideal discrete $M$-state uniform IRS. The optimal strategy for the IRS is to choose one reflection state with probability \num{1} to boost the equivalent channel, while the optimal strategy for the modulator is to utilize all constellation points with equal probability. It inspires one to adaptively design the p.m.f. of tag symbols to jointly benefit the backscatter modulation and passive beamforming.
			\end{remark}
		\end{subsection}
	\end{section}

	\bibliographystyle{IEEEtran}
	\bibliography{IEEEabrv,library.bib}
\end{document}
