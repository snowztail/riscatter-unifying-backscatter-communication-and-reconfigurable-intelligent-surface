\documentclass[final,xcolor={table}]{beamer}
%% Possible paper sizes: a0, a0b, a1, a2, a3, a4.
%% Possible orientations: portrait, landscape
%% Font sizes can be changed using the scale option.
\usepackage[size=a0,orientation=landscape]{beamerposter}

\usetheme{gemini}
\usecolortheme{seagull}
\useinnertheme{rectangles}
\usefonttheme[onlymath]{serif}


\setlength\lineskip{20pt}
% \let\OLDitemize\itemize
% \renewcommand\itemize{\OLDitemize\addtolength{\itemsep}{20pt}}
\renewcommand\baselinestretch{1.25}

% ====================
% Packages
% ====================

% \usepackage[utf8]{inputenc}
% \usepackage[table]{xcolor}
\usepackage{adjustbox}
\usepackage{graphicx}
\usepackage{booktabs}
\usepackage{tikz}
\usepackage{pgfplots}
\usepackage[T1]{fontenc}
\usepackage[sfdefault]{FiraSans}
\usepackage[acronym]{glossaries-extra}
\usepackage{siunitx}
\usepackage[caption=false,font=footnotesize,subrefformat=parens,labelformat=parens]{subfig}
\usepackage{tabularx}
\usepackage[american]{circuitikz}
\usepackage[short]{optidef}

\usetikzlibrary{arrows,calc,matrix,patterns,plotmarks,positioning,shapes}
\usetikzlibrary{decorations.pathmorphing,decorations.pathreplacing,decorations.shapes,shapes.geometric}
\usepgfplotslibrary{groupplots,patchplots}
\pgfplotsset{compat=newest}



\glsdisablehyper
\setabbreviationstyle[acronym]{long-short}
\newacronym{af}{AF}{Amplify-and-Forward}
\newacronym{ambc}{AmBC}{Ambient Backscatter Communication}
\newacronym{ap}{AP}{Access Point}
\newacronym{awgn}{AWGN}{Additive White Gaussian Noise}
\newacronym{bcd}{BCD}{Block Coordinate Descent}
\newacronym{bc}{BackCom}{Backscatter Communication}
\newacronym{bibo}{BIBO}{Binary-Input Binary-Output}
\newacronym{bpcu}{\si{bpcu}}{bits per channel use}
\newacronym{bpsphz}{\si{bps/Hz}}{bits per second per Hertz}
\newacronym{clt}{CLT}{Central Limit Theorem}
\newacronym{cw}{CW}{Continuous Waveform}
\newacronym{cp}{CP}{Canonical Polyadic}
\newacronym{cr}{CR}{Cognitive Radio}
\newacronym{cscg}{CSCG}{Circularly Symmetric Complex Gaussian}
\newacronym{csi}{CSI}{Channel State Information}
\newacronym{dc}{DC}{Direct Current}
\newacronym{df}{DF}{Decode-and-Forward}
\newacronym{dmc}{DMC}{Discrete Memoryless Channel}
\newacronym{dmtc}{DMTC}{Discrete Memoryless Thresholding Channel}
\newacronym{dmmac}{DMMAC}{Discrete Memoryless Multiple Access Channel}
\newacronym{dcmc}{DCMC}{Discrete-input Continuous-output Memoryless Channel}
\newacronym{dp}{DP}{Dynamic Programming}
\newacronym{fdma}{FDMA}{Frequency-Division Multiple Access}
\newacronym{iid}{i.i.d.}{independent and identically distributed}
\newacronym{ioe}{IoE}{Internet of Everything}
\newacronym{iot}{IoT}{Internet of Things}
\newacronym{kkt}{KKT}{Karush-Kuhn-Tucker}
\newacronym{m2m}{M2M}{Machine to Machine}
\newacronym{mac}{MAC}{Multiple Access Channel}
\newacronym{mc}{MC}{Multiplication Coding}
\newacronym{miso}{MISO}{Multiple-Input Single-Output}
\newacronym{mimo}{MIMO}{Multiple-Input Multiple-Output}
\newacronym{ml}{ML}{Maximum-Likelihood}
\newacronym{mrt}{MRT}{Maximum Ratio Transmission}
\newacronym{noma}{NOMA}{Non-Orthogonal Multiple Access}
\newacronym{ofdm}{OFDM}{Orthogonal Frequency-Division Multiplexing}
\newacronym{pdf}{PDF}{Probability Density Function}
\newacronym{pga}{PGA}{Projected Gradient Ascent}
\newacronym{psk}{PSK}{Phase Shift Keying}
\newacronym{qam}{QAM}{Quadrature Amplitude Modulation}
\newacronym{qos}{QoS}{Quality of Service}
\newacronym{rf}{RF}{Radio-Frequency}
\newacronym{rfid}{RFID}{Radio-Frequency Identification}
\newacronym{ris}{RIS}{Reconfigurable Intelligent Surface}
\newacronym{sc}{SC}{Superposition Coding}
\newacronym{sic}{SIC}{Successive Interference Cancellation}
\newacronym{simo}{SIMO}{Single-Input Multiple-Output}
\newacronym{sinr}{SINR}{Signal-to-Interference-plus-Noise Ratio}
\newacronym{smawk}{SMAWK}{Shor-Moran-Aggarwal-Wilber-Klawe}
\newacronym{snr}{SNR}{Signal-to-Noise Ratio}
\newacronym{sr}{SR}{Symbiotic Radio}
\newacronym{swipt}{SWIPT}{Simultaneous Wireless Information and Power Transfer}
\newacronym{tdma}{TDMA}{Time-Division Multiple Access}
\newacronym{ue}{UE}{user}
\newacronym{wit}{WIT}{Wireless Information Transfer}
\newacronym{wpcn}{WPCN}{Wireless Powered Communication Network}
\newacronym{wpt}{WPT}{Wireless Power Transfer}
\newacronym{mbc}{MBC}{Monostatic \glsentryshort{bc}}
\newacronym{bbc}{BBC}{Bistatic \glsentryshort{bc}}
\newacronym{bls}{BLS}{Backtracking Line Search}
\newacronym{mrc}{MRC}{Maximal Ratio Combining}
\newacronym{sdma}{SDMA}{Space-Division Multiple Access}
\newacronym{nlos}{NLoS}{Non-Line-of-Sight}
\newacronym{zf}{ZF}{Zero-Forcing}
\newacronym{mmse}{MMSE}{Minimum Mean-Square-Error}
\newacronym{fpga}{FPGA}{Field-Programmable Gate Array}
\newacronym{ber}{BER}{Bit Error Rate}



% ====================
% Lengths
% ====================

% If you have N columns, choose \sepwidth and \colwidth such that
% (N+1)*\sepwidth + N*\colwidth = \paperwidth
\newlength{\sepwidth}
\newlength{\colwidth}
\setlength{\sepwidth}{0.025\paperwidth}
\setlength{\colwidth}{0.3\paperwidth}
% \setlength{\sepwidth}{0.03\paperwidth}
% \setlength{\colwidth}{0.45\paperwidth}

\newcommand{\separatorcolumn}{\begin{column}{\sepwidth}\end{column}}

% ====================
% Logo (optional)
% ====================

% LaTeX logo taken from https://commons.wikimedia.org/wiki/File:LaTeX_logo.svg
% use this to include logos on the left and/or right side of the header:
\logoleft{\includegraphics[height=4cm]{../assets/poster/icl.eps}}
\logoright{\includegraphics[height=4cm]{../assets/poster/qr.pdf}}

% ====================
% Footer (optional)
% ====================

\footercontent{
	SAL 6G Symposium, Linz, Austria \hfill
	November 22 \& 23, 2023 \hfill
	\href{mailto:yang.zhao18@imperial.ac.uk}{\texttt{yang.zhao18@imperial.ac.uk}}
}
% (can be left out to remove footer)

% ====================
% My own customization
% - BibLaTeX
% - Boxes with tcolorbox
% - User-defined commands
% ====================
% ====================
% BibLaTeX
% ====================

% \usepackage[backend=biber,
% 	bibstyle=authoryear,
% 	citestyle=authoryear,
% 	style=authoryear,
% 	maxcitenames=2,
% 	maxbibnames=20, % limit the length of list of names (authors/editors/etc.)
% 	sorting=ydnt, % sort references by year (descending), name, title
% 	dashed=false, % show authors instead of dash in publications having the same authors
% 	giveninits=true % render authors' given name initials and not the full given names
% ]{biblatex}
% %% Biblatex with Beamer bibliography icons
% \setbeamertemplate{bibliography item}{%
% 	\ifboolexpr{ test {\ifentrytype{book}} or test {\ifentrytype{mvbook}}
% 		or test {\ifentrytype{collection}} or test {\ifentrytype{mvcollection}}
% 		or test {\ifentrytype{reference}} or test {\ifentrytype{mvreference}} }
% 	{\setbeamertemplate{bibliography item}[book]}
% 	{\ifentrytype{online}
% 		{\setbeamertemplate{bibliography item}[online]}
% 		{\setbeamertemplate{bibliography item}[article]}}%
% 	\usebeamertemplate{bibliography item}}
% \defbibenvironment{bibliography}
% {\list{}
% 	{\settowidth{\labelwidth}{\usebeamertemplate{bibliography item}}%
% 		\setlength{\leftmargin}{\labelwidth}%
% 		\setlength{\labelsep}{\biblabelsep}%
% 		\addtolength{\leftmargin}{\labelsep}%
% 		\setlength{\itemsep}{\bibitemsep}%
% 		\setlength{\parsep}{\bibparsep}}}
% {\endlist}
% {\item}
% %% Redefine \refname
% \renewcommand{\bibname}{References}
% %% Redefine \parencite to use square brackets instead of braces
% \DeclareCiteCommand{\parencite}
% {\usebibmacro{prenote}}
% {\usebibmacro{citeindex}%
% 	\printtext[bibhyperref]{[\usebibmacro{cite}]}}
% {\multicitedelim}
% {\usebibmacro{postnote}}
% %% Highlight author names using Beamer data annotation
% %% Usage: add a new line `author+an = {<author-order>=highlight}` to an entry
% %% For example: author+an = {3=highlight} => highlight the 3rd author name
% \AtBeginBibliography{
% 	\renewcommand*{\mkbibnamegiven}[1]{%
% 		\ifitemannotation{highlight}
% 		{\textbf{#1}}
% 		{#1}%
% 	}

% 	\renewcommand*{\mkbibnamefamily}[1]{%
% 		\ifitemannotation{highlight}
% 		{\textbf{#1}}
% 		{#1}%
% 	}
% }

% ====================
% Boxes with tcolorbox
% ====================
\usepackage[most]{tcolorbox}

%%% Beamer colors in boxes

\newcommand{\beamercolorsinboxes}[1]{
	\setbeamercolor{itemize item}{fg=#1!75!black}
	\setbeamercolor{itemize/enumerate body}{fg=#1!65!white}
	\setbeamercolor{itemize/enumerate subbody}{fg=#1!65!white}
	\setbeamercolor{item projected}{fg=white, bg=#1!75!black}
}

%%% Highlight Oval Box
\newtcbox{\xmybox}[1][red]{on line,
	arc=7pt,colback=#1!10!white,colframe=#1!50!black,
	before upper={\rule[-3pt]{0pt}{10pt}},boxrule=1pt,
	boxsep=0pt,left=6pt,right=6pt,top=2pt,bottom=2pt}
%%% Box for stating problems
%%%%%%%%
%Usage: (similar for infobox)
%	\begin{defbox}{title}
%		contents
%	\end{defbox}
%%%%%%%%
\newtcolorbox{defbox}[1]{%
	enhanced,
	attach boxed title to top 	left={xshift=5mm,yshift=-5mm,yshifttext=-5mm},
	colback=cyan!5!white,
	colframe=cyan!75!black,
	coltitle=cyan!80!black,
	%	left=0mm,right=0mm,top=2mm,bottom=0mm,
	title={#1},
	fonttitle=\bfseries\large, fontupper=\color{cyan!65!white},
	boxed title style={colback=cyan!5!white,colframe=cyan!75!black},
	before upper={
			\beamercolorsinboxes{cyan}
		}
}%
%%% Box for announcement
\newtcolorbox{infobox}[1]{%
	enhanced,
	attach boxed title to top 	left={xshift=5mm,yshift=-5mm,yshifttext=-5mm},
	colback=yellow,
	colframe=red!75!black,
	coltitle=red!75!black,
	%	left=0mm,right=0mm,top=2mm,bottom=0mm,
	title={#1},
	fonttitle=\bfseries\large, fontupper=\color{red!65!white},
	boxed title style={colback=yellow,colframe=red!75!black},
	before upper={
			\beamercolorsinboxes{red}
		}
}%
%%% Box for example
\newtcolorbox{exabox}[1]{%
	enhanced,
	attach boxed title to top 	left={xshift=5mm,yshift=-5mm,yshifttext=-5mm},
	colframe=brown!75!black,colback=brown!5!white,coltitle=brown!50!brown!75!black,
	%	left=0mm,right=0mm,top=2mm,bottom=0mm,
	title={#1},
	fonttitle=\bfseries\large, fontupper=\color{brown!65!white},
	boxed title style={colback=brown!5!white,coltitle=brown!50!brown!75!black},
	before upper={
			\beamercolorsinboxes{brown}
		}
}%
%%% Theorem Box
%%%%%%%%
%Usage: (similar for conjecture, lemma, etc.)
%	\begin{thm}{title}{nameref}
%		contents
%	\end{thm}
% Use \ref{thm:nameref} to refer to the theorem
%%%%%%%%
%%%% Use \newtcbtheorem[number within=section]{thm} to number within each section
\newtcbtheorem[]{thm}%
% {Theorem}{attach boxed title to top 	left={xshift=5mm,yshift=-5mm,yshifttext=-5mm},
{Block}{attach boxed title to top 	left={xshift=5mm,yshift=-5mm,yshifttext=-5mm},
	enhanced jigsaw,
	top=2mm,bottom=6mm,left=0mm,right=0mm,
	% fonttitle=\bfseries\large,fontupper=\itshape\color{blue!65!white},
	fonttitle=\bfseries\large,fontupper=\color{blue!65!white},
	colframe=blue!75!black,colback=blue!5!white,coltitle=blue!50!blue!75!black,
	boxed title style={colback=blue!5!white,coltitle=blue!50!blue!75!black},
	before upper={
			\beamercolorsinboxes{blue}
		}
}{thm}%
%%% Proposition Box
\newtcbtheorem[use counter from=thm]{prop}%
{Proposition}{attach boxed title to top 	left={xshift=5mm,yshift=-5mm,yshifttext=-5mm},
	enhanced jigsaw,
	%	top=2mm,bottom=0mm,left=0mm,right=0mm,
	fonttitle=\bfseries\large,fontupper=\itshape,
	colframe=gray!75!black,colback=gray!5!white,coltitle=gray!50!gray!75!black,
	boxed title style={colback=gray!5!white,coltitle=gray!50!gray!75!black},
	before upper={
			\beamercolorsinboxes{gray}
		}
}{prop}%
%%% Conjecture Box
\newtcbtheorem[use counter from=thm]{conj}%
{Conjecture}{attach boxed title to top 	left={xshift=5mm,yshift=-5mm,yshifttext=-5mm},
	enhanced jigsaw,
	%	top=2mm,bottom=0mm,left=0mm,right=0mm,
	fonttitle=\bfseries\large,fontupper=\slshape,
	colframe=orange!75!black,colback=orange!5!white,coltitle=orange!50!orange!75!black,
	boxed title style={colback=orange!5!white,coltitle=orange!50!orange!75!black},
	before upper={
			\beamercolorsinboxes{orange}
		}
}{conj}%
%%% Lemma Box
\newtcbtheorem[use counter from=thm]{lem}%
{Lemma}{attach boxed title to top 	left={xshift=5mm,yshift=-5mm,yshifttext=-5mm},
	enhanced jigsaw,
	%	top=2mm,bottom=0mm,left=0mm,right=0mm,
	fonttitle=\bfseries\large,fontupper=\itshape,
	colframe=green!75!black,colback=green!5!white,coltitle=green!50!green!75!black,
	boxed title style={colback=green!5!white,coltitle=green!50!green!75!black},
	before upper={
			\beamercolorsinboxes{green}
		}
}{lem}%
%%% Claim Box
\newtcbtheorem[use counter from=thm]{clm}%
{Claim}{attach boxed title to top 	left={xshift=5mm,yshift=-5mm,yshifttext=-5mm},
	enhanced jigsaw,
	%	top=2mm,bottom=0mm,left=0mm,right=0mm,
	fonttitle=\bfseries\large,fontupper=\itshape,
	colframe=pink!75!black,colback=pink!5!white,coltitle=pink!50!pink!75!black,
	boxed title style={colback=pink!5!white,coltitle=pink!50!pink!75!black},
	before upper={
			\beamercolorsinboxes{pink}
		}
}{clm}%


%% Reference Sources
% \addbibresource{library.bib}
\renewcommand{\pgfuseimage}[1]{\includegraphics[scale=2.0]{#1}}

\title{RIScatter: Unifying BackCom and RIS via Probabilistic Input Design}

\author{Yang Zhao \and Bruno Clerckx}

\date{January 01, 2025}

\begin{document}

\begin{frame}[t]
	\begin{columns}[t]
		\separatorcolumn

		\begin{column}{\colwidth}
			\begin{block}{Overview}
				\begin{itemize}\setlength\itemsep{20pt}
					\item \textbf{What does this paper propose?}

					RIScatter -- a batteryless cognitive radio that recycles ambient signal in an adaptive and customizable manner.
					\item \textbf{How does it differ from previous work?}

					Backscatter modulation and passive beamforming are seamlessly integrated from the perspective of probability distribution.
					\item \textbf{What are the benefits?}

					It supports cooperative and distributed deployment, avoids complex architecture and signal processing, and can be built over legacy systems.
				\end{itemize}
			\end{block}

			\begin{block}{RIScatter system}
				\begin{figure}[!t]
					\centering
					\resizebox{0.5\linewidth}{!}{
						\begin{tikzpicture}[font=\LARGE,every node/.style={draw,ultra thick},every path/.style={ultra thick},every text node part/.style={align=center}]
	\node at (0,0) [circular sector,shape border rotate=90,minimum width=2cm] {};
	\node at (5,3) [circle,minimum size=1.25cm] {};
	\node at (8,0.55) [circular sector,shape border rotate=270,minimum width=2cm] {};

	\node[draw=none] at (0,-1.25) {Primary\\Transmitter};
	\node[draw=none] at (5,4.5) {RIScatter\\Node};
	\node[draw=none] at (8,-1.25) {Co-located\\Receiver};

	\draw[blue,decorate,decoration={waves,segment length=4mm,radius=2mm}] (1,-0.25) -- (7.375,-0.25);
	\draw[blue,decorate,decoration={waves,segment length=4mm,radius=2mm}] (0.5,0.75) -- (4,3);
	\draw[blue,decorate,decoration={waves,segment length=4mm,radius=2mm}] (6,3.1625) -- (8,1.25);
	\draw[green,decorate,decoration={crosses,segment length=8mm,shape size=1.5mm}] (6.29,2.885) -- (8,1.25);

	\node[draw=none] at (3.5,0.75) {Modulated};
	\node[draw=none] at (8,2.76) {Flexible};
\end{tikzpicture}

					}
					\label{fg:riscatter}
				\end{figure}
				\begin{itemize}\setlength\itemsep{20pt}
					\item {\color{blue}Primary link:} active ambient transmission from an RF source
					\item {\color{red}Backscatter link:} passive free-ride transmission from IoT nodes
				\end{itemize}
			\end{block}

			\begin{block}{Node architecture}
				\begin{figure}[!t]
					\captionsetup[subfloat]{captionskip=20pt}
					\centering
					\subfloat[Block Diagram]{
						\resizebox{0.53\linewidth}{!}{
							\makeatletter
\tikzset{
	block/.style={draw,rectangle,align=center,minimum width=4cm,minimum height=1cm},
	from/.style args={#1 to #2}{
			above right={0cm of #1},
			/utils/exec=\pgfpointdiff
			{\tikz@scan@one@point\pgfutil@firstofone(#1)\relax}
			{\tikz@scan@one@point\pgfutil@firstofone(#2)\relax},
			% minimum width/.expanded=\the\pgf@x,
			% minimum height/.expanded=\the\pgf@y
		}
}
\makeatother

\begin{circuitikz}[transform shape]
	\tikzstyle{every node}=[font=\tiny]
	\coordinate (O) at (0,0);
	% \node[block,from={O to $(O) + (7,4)$}](T){};
	\draw (0,1.5)
	to[short] ++(-1,0)
	to[short] ++(0,1) node[bareantenna](A){Sx};
	\draw (0,1.5)
	to[short,-*] ++(0.25,0) coordinate(J);
	\draw (J)
	to[short] ++(0,1.5)
	to[short] ++(0.5,0) coordinate(J1);
	\node[block,from={$(J1) + (0,-0.5)$ to $(J1) + (2.5,0.5)$}](R){Rectifier};
	\draw (J)
	to[short] ++(0.5,0) coordinate(J2);
	\node[block,from={$(J2) + (0,-0.5)$ to $(J2) + (2.5,0.5)$}](D){Demodulator};
	\draw (J)
	to[short] ++(0,-1.5)
	to[short] ++(0.5,0) coordinate(J3);
	\node[block,from={$(J3) + (0,-0.5)$ to $(J3) + (2.5,0.5)$}](M){Modulator};
	\draw[dashed,-{Latex[length=2mm]}] (R.east) -- ++(0.75,0);
	\draw[-{Latex[length=2mm]}] (D.east) -- ++(0.75,0);
	\draw[{Latex[length=2mm]}-] (M.east) -- ++(0.75,0);
	\node[block,from={$(R.east) + (0.75,-0.5)$ to $(R.east) + (2.875,0.5)$}](P){Power Buffer};
	\node[block,minimum height=2.5cm,from={$(M.east) + (0.75,-0.5)$ to $(D.east) + (2.875,0.5)$}](S){Digital Section};
	\draw[dashed,-{Latex[length=2mm]}] (P.south) to (S.north);
	\coordinate (F1) at ($(P.south)!0.5!(S.north)$);
	\coordinate (F2) at ($(D.east)!0.5!(S.west)$);
	\coordinate (F3) at ($(D.south)!0.5!(M.north)$);
	\draw[dashed] (F1) to (F1-|D.north);
	\draw[dashed,-{Latex[length=2mm]}] (F1-|D.north) to (D.north);
	\draw[dashed] (F1-|F2) to (F2|-F3) to (M|-F3);
	\draw[dashed,-{Latex[length=2mm]}] (M|-F3) to (M.north);
\end{circuitikz}

						}
						\label{fg:block_diagram}
					}
					\subfloat[Scatter Model]{
						\resizebox{0.42\linewidth}{!}{
							\begin{circuitikz}[transform shape]
	\tikzstyle{every node}=[font=\Large]
	\draw (0,0) node[bareantenna](bareantenna){};
	\draw (bareantenna.west) ++(-1.5,0) node[waves](WI){};
	\draw (WI.north east) ++(0.25,0) node[font=\Large]{$\vec{E}_{\mathrm{I}}$};
	\draw (WI.south east) ++(0.25,0) node[font=\Large]{$\vec{H}_{\mathrm{I}}$};
	\draw (bareantenna.east) ++(0.875,0) node[waves](WR){};
	\draw (WR.north east) ++(0.35,0) node[font=\Large]{$\vec{E}_m$};
	\draw (WR.south east) ++(0.35,0) node[font=\Large]{$\vec{H}_m$};
	\draw (bareantenna)
	to [R=$Z_{\mathrm{A}}$] ++(0,-1.5) node[rotary switch=4 in 90 wiper 30,anchor=ext center,rotate=270](SW){};
	\draw (SW.cout 2) node[below=0.1cm,font=\Large]{$m$}
	to [R=$Z_m$] ++(2,0) node[ground]{};
\end{circuitikz}

						}
						\label{fg:scatter_model}
					}
					\label{fg:riscatter_node}
				\end{figure}
				\vspace{0.62cm}
				\begin{itemize}\setlength\itemsep{20pt}
					\item Wave scattering or reflection are manipulated by antenna or metamaterial
					\item Incoming signals are used for powering, modulation, and beamforming
					\item The node changes reflection state by switching load impedance
					% \item A reflection state corresponds to a load impedance
				\end{itemize}
			\end{block}

			\begin{alertblock}{Properties}
				\begin{enumerate}\setlength\itemsep{20pt}
					\item Primary and backscatter symbols are superimposed by \emph{double modulation}
					\item Backscatter signal is much weaker due to \emph{double fading}
					\item The spreading factor (symbol period ratio) is usually large
					\item Each \emph{state} is simultaneously part of information and beamforming \emph{codeword}
					\item Reflection pattern over time is semi-random and guided by input probability assigned to each state
				\end{enumerate}
			\end{alertblock}
		\end{column}

		\separatorcolumn

		\begin{column}{2\colwidth+\sepwidth}
			\begin{block}{Applications comparison}
				\begin{table}[!t]
					\rowcolors{2}{gray!25}{white}
					\renewcommand{\arraystretch}{1.25}
					\begin{tabular}{c c c c c c}
						\toprule
						\hiderowcolors
						                       & Backscatter        & Ambient backscatter         & Symbiotic radio          & Reconfigurable intelligent surface & RIScatter                                          \\ \midrule
						\showrowcolors
						Information link(s)    & Backscatter        & Coexisting                  & Coexisting               & Primary                            & Coexisting                                         \\
						Primary on backscatter & Carrier            & Multiplicative interference & Spreading code           & ---                                & Energy uncertainty                                 \\
						Backscatter on primary & ---                & Multiplicative interference & Channel uncertainty      & Passive beamforming                & Dynamic passive beamforming                        \\
						Cooperative devices    & ---                & No                          & Transmitter and receiver & ---                                & Transmitter, nodes, and receiver                   \\
						Sequential decoding    & ---                & No                          & Primary-to-backscatter   & ---                                & Backscatter-to-primary                             \\
						Reflection pattern by  & Information source & Information source          & Information source       & Channel                            & Information source, channel, and relative priority \\
						Input distribution     & Equiprobable       & Equiprobable                & Equiprobable or Gaussian & Degenerate                         & Flexible                                           \\
						Load-switching speed   & Fast               & Slow                        & Slow                     & Quasi-static                       & Arbitrary                                          \\ \bottomrule
					\end{tabular}
				\end{table}
			\end{block}

			\vspace{-1.3cm}
			\begin{columns}[t,totalwidth=\textwidth]
				\begin{column}{\colwidth}
					\begin{block}{Low-complexity receiver}
						\begin{figure}[!t]
							\centering
							\subfloat{
								\resizebox{0.42\linewidth}{!}{
									\begin{tikzpicture}[
		% every axis plot/.append style={thick},
		every node/.append style={draw,minimum width=16cm},
		align=center
	]
	\draw[-{Latex[length=4mm]}] (0,0) node[anchor=south](0){Semi-coherently decode node message} to ++(0,-1) node[anchor=north](1){Re-encode for reflection pattern};
	\draw[-{Latex[length=4mm]}] (1.south) to ++(0,-1) node[anchor=north](2){Model within equivalent channel};
	\draw[-{Latex[length=4mm]}] (2.south) to ++(0,-1) node[anchor=north](3){Coherently decode primary message};
\end{tikzpicture}

								}
							}
							\hspace{-1.5cm}
							\subfloat{
								\resizebox{0.58\linewidth}{!}{
									\begin{tikzpicture}
	\begin{axis}[%
			height=5cm,
			width=9cm,
			xlabel = $z$,
			ylabel = {Probability Density},
			font=\scriptsize,
			no markers,
			xmin=0,
			ymin=0,
			xmax=70,
			%     ticks=none,
			xtick={0,12.78,20.28,29.62,70},
			xticklabels={$t_0$,$t_1$,$t_2$,$t_3$,$t_4$},
			yticklabels={},
			samples = 200]

		\addplot+[thick] gnuplot[raw gnuplot] {%
				isint(x) = (int(x)==x);
				gmm(x,rho,lambda)=rho<=0||lambda<=0?1/0:  x<0?0.0:x==0?(rho>1?0.0:rho==1?real(lambda):1/0):  exp(rho*log(lambda)+(rho-1.0)*log(x)-lgamma(rho)-lambda*x);
				set xrange [0:80];
				set yrange [0:1];
				samples=200;
				plot gmm(x,10,1)};
		\addlegendentryexpanded{$f(z \mid \mathcal{H}_1)$}

		\addplot+[thick] gnuplot[raw gnuplot] {%
				isint(x) = (int(x)==x);
				gmm(x,rho,lambda)=rho<=0||lambda<=0?1/0:  x<0?0.0:x==0?(rho>1?0.0:rho==1?real(lambda):1/0):  exp(rho*log(lambda)+(rho-1.0)*log(x)-lgamma(rho)-lambda*x);
				set xrange [0:80];
				set yrange [0:1];
				samples=200;
				plot gmm(x,10,0.6)};
		\addlegendentryexpanded{$f(z \mid \mathcal{H}_2)$}

		\addplot+[thick,color=brown] gnuplot[raw gnuplot] {%
				isint(x) = (int(x)==x);
				gmm(x,rho,lambda)=rho<=0||lambda<=0?1/0:  x<0?0.0:x==0?(rho>1?0.0:rho==1?real(lambda):1/0):  exp(rho*log(lambda)+(rho-1.0)*log(x)-lgamma(rho)-lambda*x);
				set xrange [0:80];
				set yrange [0:1];
				samples=200;
				plot gmm(x,10,0.4)};
		\addlegendentryexpanded{$f(z \mid \mathcal{H}_3)$}

		\addplot+[thick,color=green!40!gray] gnuplot[raw gnuplot] {%
				isint(x) = (int(x)==x);
				gmm(x,rho,lambda)=rho<=0||lambda<=0?1/0:  x<0?0.0:x==0?(rho>1?0.0:rho==1?real(lambda):1/0):  exp(rho*log(lambda)+(rho-1.0)*log(x)-lgamma(rho)-lambda*x);
				set xrange [0:80];
				set yrange [0:1];
				samples=200;
				plot gmm(x,10,0.28)};
		\addlegendentryexpanded{$f(z \mid \mathcal{H}_4)$}
		\draw[color=gray,thick,dashed] (12.78,0) -- (12.78,0.1);
		\draw[color=gray,thick,dashed] (20.28,0) -- (20.28,0.1);
		\draw[color=gray,thick,dashed] (29.62,0) -- (29.62,0.1);
		\draw[latex-latex] (0,0.05) -- (12.78,0.05) node[below,midway]{$\mathcal{R}_1$};
		\draw[latex-latex] (12.78,0.05) -- (20.28,0.05) node[below,midway]{$\mathcal{R}_2$};
		\draw[latex-latex] (20.28,0.05) -- (29.62,0.05) node[below,midway]{$\mathcal{R}_3$};
		\draw[latex-latex] (29.62,0.05) -- (70,0.05) node[below,midway]{$\mathcal{R}_4$};
	\end{axis}
\end{tikzpicture}

								}
							}
							\label{fg:receiver}
						\end{figure}
						\vspace{0.5cm}
						\begin{itemize}\setlength\itemsep{20pt}
							\item Accumulated receive energy follows conditional Gamma distribution
							\item Node detection under primary uncertainty becomes part of channel training
							\item Requires one additional energy comparison and re-encoding per backscatter symbol
							% \item Backscatter achievable rate depends on decision region design
						\end{itemize}
					\end{block}

					\begin{block}{Problem formulation}
						\vspace{-0.5cm}
						\begin{columns}[c,totalwidth=\textwidth]
							\begin{column}{0.48\colwidth}
								\begin{maxi*}
									{\scriptstyle{\{\boldsymbol{p}_k\},\boldsymbol{w},\boldsymbol{t}}}{\rho R_\text{P} + (1-\rho) \sum_k R_{\text{B},k}}{}{}
									\addConstraint{\boldsymbol{1}^\mathsf{T} \boldsymbol{p}_k=1,}{\quad \boldsymbol{p}_k \ge \boldsymbol{0},}{\quad \forall k}
									\addConstraint{t_{l-1} \le t_l,}{\quad t_l \ge 0,}{\quad \forall l}
									\addConstraint{\lVert \boldsymbol{w} \rVert^2 \le P}{}{}
								\end{maxi*}
							\end{column}
							\begin{column}{0.48\colwidth}
								\begin{itemize}\setlength\itemsep{20pt}
									% \item $\rho$ is the priority of the primary link
									\item $\boldsymbol{p}_k$ is the input distribution of node $k$
									\item $\boldsymbol{w}$ is the active beamforming vector
									\item $\boldsymbol{t}$ is the decision threshold vector
								\end{itemize}
							\end{column}
						\end{columns}
					\end{block}

					\begin{thm}{Input distribution}{}
						\begin{equation*}
							p^{(r+1)}(x_{m_k}) \gets \frac{p^{(r)}(x_{m_k}) \exp \Bigl( \frac{\rho}{1 - \rho} I_k^{(r)}(x_{m_k}) \Bigr)}{\sum_{m_k'} p^{(r)}(x_{m_k'}) \exp \Bigl( \frac{\rho}{1 - \rho} I_k^{(r)}(x_{m_k'}) \Bigr)}
							\label{eq:input_kkt_solution}
						\end{equation*}
					\end{thm}

					\begin{thm}{Decision threshold}{}
						\begin{figure}[!t]
							\centering
							\resizebox{0.65\linewidth}{!}{
								\begin{tikzpicture}
	\begin{axis}[width=10cm,height=5cm,xmin=0,xmax=6.5,xlabel=$z$,axis x line=middle,axis y line=none,xtick={0,2,3,4,6},xticklabels={$t_0$,$t_1$,$t_2$,$t_3$,$t_4$},xtick style={draw=none},every axis x label/.style={at=(current axis.right of origin),anchor=west},domain=0:6]
		\addplot[mark=*,mark size=0.5pt,only marks,color=gray] {0};
		\addplot[mark=*,mark size=1pt,only marks,color=black] coordinates{(0,0) (2,0) (3,0) (4,0) (6,0)};
		\draw[decorate,decoration={brace,amplitude=10}] (0,0) -- (2,0) node[above,midway,yshift=10]{$\mathcal{R}_1$};
		\draw[decorate,decoration={brace,amplitude=10}] (2,0) -- (3,0) node[above,midway,yshift=10]{$\mathcal{R}_2$};
		\draw[decorate,decoration={brace,amplitude=10}] (3,0) -- (4,0) node[above,midway,yshift=10]{$\mathcal{R}_3$};
		\draw[decorate,decoration={brace,amplitude=10}] (4,0) -- (6,0) node[above,midway,yshift=10]{$\mathcal{R}_4$};
	\end{axis}
\end{tikzpicture}

							}
						\end{figure}
						\begin{itemize}\setlength\itemsep{15pt}
							% \item Constrain the domain to fine-grained energy levels
							\item Obtain the rate-optimal quantization by dynamic programming or bisection
						\end{itemize}
					\end{thm}

					\begin{thm}{Active beamformer}{}
						\begin{equation*}
							\boldsymbol{w}^{(r+1)} \gets \mathrm{proj}_{\lVert \boldsymbol{w} \rVert^2 \le P}\Bigl(\boldsymbol{w}^{(r)}+\gamma\nabla_{\boldsymbol{w}^*} R^{(r)}\Bigr),
							\label{eq:beamforming_gradient_ascent}
						\end{equation*}
					\end{thm}
				\end{column}

				\separatorcolumn
				\begin{column}{\colwidth}
					\begin{exampleblock}{Input distribution}
						\begin{figure}[!t]
							\centering
							\subfloat{
								\resizebox{0.48\linewidth}{!}{
									\begin{tikzpicture}[font=\tiny]
	\begin{groupplot}
		[group style={
					rows=3,
					group name=plots,
					x descriptions at=edge bottom,
					y descriptions at=edge left,
					vertical sep=10
				},
			ybar,
			xmin=0,
			xmax=4,
			xtick={1,2,3,4},
			ymin=0,
			ymax=1,
			xlabel={Reflection State},
			enlarge x limits={value=0.25,upper},
			enlarge y limits={value=0.1,upper},
			width=11.6cm,
			height=3cm,
			x tick style={draw=none},
			x label style={inner sep=5.5pt},
			% font=\tiny,
			legend style={fill=none},
			every axis plot/.append style={bar shift=0}
		]
		\nextgroupplot
		\addplot[blue,fill=blue!30!white] coordinates {(-1,-1)};
		\addplot[blue,fill=blue!30!white,postaction={pattern=north west lines},pattern color=.] coordinates {(1,0.25)};
		\addplot[blue,fill=blue!30!white,postaction={pattern=north east lines},pattern color=.] coordinates {(2,0.25)};
		\addplot[blue,fill=blue!30!white,postaction={pattern=vertical lines},pattern color=.] coordinates {(3,0.25)};
		\addplot[blue,fill=blue!30!white,postaction={pattern=horizontal lines},pattern color=.] coordinates {(4,0.25)};
		\legend{BackCom}
		\nextgroupplot[ylabel={Probability Distribution}]
		\addplot[red,fill=red!30!white] coordinates {(-1,-1)};
		\addplot[red,fill=red!30!white,postaction={pattern=north west lines},pattern color=.] coordinates {(1,0)};
		\addplot[red,fill=red!30!white,postaction={pattern=north east lines},pattern color=.] coordinates {(2,1)};
		\addplot[red,fill=red!30!white,postaction={pattern=vertical lines},pattern color=.] coordinates {(3,0)};
		\addplot[red,fill=red!30!white,postaction={pattern=horizontal lines},pattern color=.] coordinates {(4,0)};
		\legend{RIS}
		\nextgroupplot
		\addplot[brown,fill=brown!30!white] coordinates {(-1,-1)};
		\addplot[brown,fill=brown!30!white,postaction={pattern=north west lines},pattern color=.] coordinates {(1,0.25)};
		\addplot[brown,fill=brown!30!white,postaction={pattern=north east lines},pattern color=.] coordinates {(2,0.75)};
		\addplot[brown,fill=brown!30!white,postaction={pattern=vertical lines},pattern color=.] coordinates {(3,0)};
		\addplot[brown,fill=brown!30!white,postaction={pattern=horizontal lines},pattern color=.] coordinates {(4,0)};
		\legend{RIScatter}
	\end{groupplot}
\end{tikzpicture}

								}
							}
							\subfloat{
								\resizebox{0.48\linewidth}{!}{
									% This file was created by matlab2tikz.
%
%The latest updates can be retrieved from
%  http://www.mathworks.com/matlabcentral/fileexchange/22022-matlab2tikz-matlab2tikz
%where you can also make suggestions and rate matlab2tikz.
%
\definecolor{mycolor1}{rgb}{0.00000,0.44706,0.74118}%
\definecolor{mycolor2}{rgb}{0.85098,0.32549,0.09804}%
\definecolor{mycolor3}{rgb}{0.92941,0.69412,0.12549}%
\definecolor{mycolor4}{rgb}{0.49412,0.18431,0.55686}%
%
\begin{tikzpicture}

\begin{axis}[%
width=4.079in,
height=1.587in,
at={(0.684in,0.361in)},
scale only axis,
xmin=1,
xmax=4,
xtick={1, 2, 3, 4},
xlabel style={font=\color{white!15!black}},
xlabel={Reflection State},
ymin=0,
ymax=1,
ytick={  0, 0.2, 0.4, 0.6, 0.8,   1},
ylabel style={font=\color{white!15!black}},
ylabel={Probability\\Distribution},
axis background/.style={fill=white},
xmajorgrids,
ymajorgrids,
legend style={at={(0.03,0.97)}, anchor=north west, legend cell align=left, align=left, draw=white!15!black},
align=center,
title style={font=\LARGE},
label style={font=\LARGE},
ticklabel style={font=\Large},
legend style={font=\Large}
]
\addplot [color=mycolor1, line width=2.0pt, mark=o, mark options={solid, mycolor1}]
  table[row sep=crcr]{%
1	6.48337921881147e-07\\
2	0.504332536970115\\
3	0.495666065268481\\
4	7.49423481629768e-07\\
};
\addlegendentry{$\rho =0$}

\addplot [color=mycolor2, dashed, line width=2.0pt, mark=+, mark options={solid, mycolor2}]
  table[row sep=crcr]{%
1	9.07520883479571e-08\\
2	0.401497967775065\\
3	0.598501822229458\\
4	1.1924338950349e-07\\
};
\addlegendentry{$\rho =0.1$}

\addplot [color=mycolor3, dotted, line width=2.0pt, mark=square, mark options={solid, mycolor3}]
  table[row sep=crcr]{%
1	5.20790909800095e-09\\
2	0.192484685585801\\
3	0.807515300031504\\
4	9.17478600370393e-09\\
};
\addlegendentry{$\rho =0.25$}

\addplot [color=mycolor4, dashdotted, line width=2.0pt, mark=x, mark options={solid, mycolor4}]
  table[row sep=crcr]{%
1	1.08446044988811e-11\\
2	8.27854203489531e-06\\
3	0.999991721175464\\
4	2.71656646770891e-10\\
};
\addlegendentry{$\rho =1$}

\end{axis}
\end{tikzpicture}%
								}
							}
						\end{figure}
						Backscatter communication and reconfigurable intelligent surface are special cases of RIScatter with uniform and degenerate input distribution.
						Increasing $\rho$ from 0 to 1 creates a smooth transition from backscatter modulation to passive beamforming.
					\end{exampleblock}
					\begin{exampleblock}{Rate region}
						\begin{figure}[!t]
							\centering
							\subfloat{
								\resizebox{0.48\linewidth}{!}{
									% This file was created by matlab2tikz.
%
%The latest updates can be retrieved from
%  http://www.mathworks.com/matlabcentral/fileexchange/22022-matlab2tikz-matlab2tikz
%where you can also make suggestions and rate matlab2tikz.
%
\definecolor{mycolor1}{rgb}{0.30100,0.74500,0.93300}%
\definecolor{mycolor2}{rgb}{0.46600,0.67400,0.18800}%
\definecolor{mycolor3}{rgb}{0.49400,0.18400,0.55600}%
\definecolor{mycolor4}{rgb}{0.92900,0.69400,0.12500}%
\definecolor{mycolor5}{rgb}{0.85000,0.32500,0.09800}%
\definecolor{mycolor6}{rgb}{0.00000,0.44700,0.74100}%
%
\begin{tikzpicture}

	\begin{axis}[%
			width=10cm,
			height=5.55cm,
			at={(0.673in,0.356in)},
			scale only axis,
			xmin=0,
			xmax=6.5227548374066,
			xlabel style={font=\color{white!15!black}},
			xlabel={Primary Rate [bits/s/Hz]},
			ymin=0,
			ymax=2,
			ylabel style={font=\color{white!15!black}},
			ylabel={Backscatter Rate [bits/cu]},
			% axis background/.style={fill=white},
			xmajorgrids,
			ymajorgrids,
			legend style={at={(0.03,0.97)}, anchor=north west, legend cell align=left, align=left, draw=white!15!black, fill=none},
			align=center,
			font=\tiny,
			title style={font=\tiny},
			label style={font=\tiny},
			ticklabel style={font=\tiny},
			legend style={font=\tiny},
			reverse legend,
			every axis plot/.append style={line width=2pt}
		]
		\addplot [color=mycolor1, line width=2.0pt, mark=triangle, mark options={solid, rotate=180, mycolor1}]
		table[row sep=crcr]{%
				6.33724402501803	1.39764296268229\\
				0	1.39764296268229\\
				0	0\\
				6.52275483678685	0\\
				6.52275483678685	2.37312052924553e-08\\
				6.38739454470992	1.3235811846854\\
				6.35422040211439	1.38916612409166\\
				6.34857862205488	1.39386686170527\\
				6.3445353675389	1.39608085009443\\
				6.34291602910554	1.3966977322109\\
				6.34149846039129	1.39711118534274\\
				6.34024731759122	1.39737797077851\\
				6.3391350240222	1.3975379071887\\
				6.33862394227692	1.39758701988787\\
				6.33813974610465	1.39761939115801\\
				6.33795313341839	1.39762818964178\\
				6.33777036860279	1.39763482338317\\
				6.3375913342378	1.3976394187338\\
				6.33741590881855	1.39764209463684\\
				6.33724402501803	1.39764296268229\\
			};
		\addlegendentry{RIScatter}

		\addplot[only marks, mark=triangle, mark options={}, mark size=2.3570pt, draw=mycolor2] table[row sep=crcr]{%
				x	y\\
				6.5227548374066	0\\
			};
		\addlegendentry{RIS}

		\addplot[only marks, mark=+, mark options={}, mark size=3.5355pt, draw=mycolor3] table[row sep=crcr]{%
				x	y\\
				6.34321982467796	2\\
			};
		\addlegendentry{SR}

		\addplot[only marks, mark=x, mark options={}, mark size=3.5355pt, draw=mycolor4] table[row sep=crcr]{%
				x	y\\
				5.90778796357092	1.34857377961699\\
			};
		\addlegendentry{AmBC}

		\addplot[only marks, mark=square, mark options={}, mark size=2.5000pt, draw=mycolor5] table[row sep=crcr]{%
				x	y\\
				0	1.99999999999997\\
			};
		\addlegendentry{Backscatter-only}

		\addplot[only marks, mark=o, mark options={}, mark size=2.7386pt, draw=mycolor6] table[row sep=crcr]{%
				x	y\\
				6.34314881160129	0\\
			};
		\addlegendentry{Primary-only}

	\end{axis}
\end{tikzpicture}%

								}
							}
							\subfloat{
								\resizebox{0.48\linewidth}{!}{
									% This file was created by matlab2tikz.
%
%The latest updates can be retrieved from
%  http://www.mathworks.com/matlabcentral/fileexchange/22022-matlab2tikz-matlab2tikz
%where you can also make suggestions and rate matlab2tikz.
%
\definecolor{mycolor1}{rgb}{0.00000,0.44706,0.74118}%
\definecolor{mycolor2}{rgb}{0.85098,0.32549,0.09804}%
\definecolor{mycolor3}{rgb}{0.92941,0.69412,0.12549}%
\definecolor{mycolor4}{rgb}{0.49412,0.18431,0.55686}%
%
\begin{tikzpicture}

\begin{axis}[%
width=4.079in,
height=3.432in,
at={(0.684in,0.463in)},
scale only axis,
xmin=0,
xmax=8.8183272162648,
xlabel style={font=\color{white!15!black}},
xlabel={Primary Rate [bits/s/Hz]},
ymin=0,
ymax=0.0387033207318202,
ylabel style={font=\color{white!15!black}},
ylabel={Total Backscatter Rate [bits/PB]},
axis background/.style={fill=white},
xmajorgrids,
ymajorgrids,
legend style={at={(0.03,0.03)}, anchor=south west, legend cell align=left, align=left, draw=white!15!black},
title style={font=\huge},
label style={font=\huge},
ticklabel style={font=\LARGE},
legend style={font=\LARGE},
scaled y ticks=false,
y tick label style={/pgf/number format/.cd, fixed, precision=3}
]
\addplot [color=mycolor1, line width=2.0pt, mark=o, mark options={solid, mycolor1}]
  table[row sep=crcr]{%
1.48510320561817	0.0368420449632478\\
0	0.0368420449632478\\
0	0\\
8.81494977830283	0\\
8.81494977830283	3.83377782240587e-09\\
8.81477352076949	4.12086540411783e-05\\
8.80464308481592	0.00150570482545958\\
8.79481987635483	0.00225430896534762\\
8.77713969997226	0.00314972206304367\\
8.76137548769293	0.00370957602960849\\
8.73671299086267	0.00438218933776903\\
8.69453600558026	0.00520601203273033\\
8.28548800580697	0.0100384418588948\\
7.50094640226552	0.0164759402145025\\
6.17365035431449	0.0243306497651073\\
5.38460240649074	0.0281228393432611\\
4.43736577109357	0.0316815244629786\\
3.35788745554615	0.0347239772599961\\
2.42247216478758	0.0363829293680192\\
1.48510320561817	0.0368420449632478\\
};
\addlegendentry{$N =10$}

\addplot [color=mycolor2, dashed, line width=2.0pt, mark=+, mark options={solid, mycolor2}]
  table[row sep=crcr]{%
1.67326087069726	0.0387033207318202\\
0	0.0387033207318202\\
0	0\\
8.81632618502012	0\\
8.81632618502012	3.78776722621619e-09\\
8.81475527355384	0.000280915977176361\\
8.79137070993339	0.00206604004006254\\
8.77128743184392	0.00286700361188652\\
8.73376638330174	0.00384928202287025\\
8.70398959350056	0.00439622330702994\\
8.65744036384737	0.00503258656132342\\
8.09711868907005	0.00979758641066475\\
5.99563108181371	0.0228762637504772\\
4.45257316761243	0.0300452011722592\\
2.27166407033345	0.0382673640700557\\
2.16776341617934	0.0385860807078977\\
1.67326087069726	0.0387033207318202\\
};
\addlegendentry{$N =20$}

\addplot [color=mycolor3, dotted, line width=2.0pt, mark=square, mark options={solid, mycolor3}]
  table[row sep=crcr]{%
1.83746579456681	0.0276333556833912\\
0	0.0276333556833912\\
0	0\\
8.81770456115495	0\\
8.81770456115495	3.7090329843289e-09\\
8.81076099034007	0.000692022846241925\\
8.76252321046823	0.0026134586736102\\
8.72405262673134	0.00338528337241505\\
8.66910163543863	0.00412499790973992\\
8.63636424982091	0.00443320983809654\\
8.49104009736616	0.00532943743607748\\
6.632727264277	0.0136498550235196\\
3.97653834627339	0.022619911417612\\
2.52017820134887	0.0269919450432153\\
2.36966502343278	0.0273353671781562\\
2.26462330968326	0.0274987365111921\\
1.83746579456681	0.0276333556833912\\
};
\addlegendentry{$N =40$}

\addplot [color=mycolor4, dashdotted, line width=2.0pt, mark=x, mark options={solid, mycolor4}]
  table[row sep=crcr]{%
1.92697586132023	0.0156929341853527\\
0	0.0156929341853527\\
0	0\\
8.8183272162648	0\\
8.8183272162648	3.59260549629013e-09\\
8.79705312747263	0.0011834340984776\\
8.71449510548813	0.00289050386525025\\
8.6694989698604	0.00334507596844779\\
8.61820727629173	0.00370001890167476\\
8.58563007809167	0.00385293178365339\\
8.07998156196288	0.00534682829297022\\
5.73983931093962	0.0105561822763253\\
4.02882077928576	0.0135099872091105\\
3.51115343811173	0.0143127975860811\\
3.14578535673428	0.0148283101617506\\
2.91655857645647	0.0151149267525588\\
2.72534244455508	0.0153412126648229\\
2.51360524723683	0.0155222047472113\\
2.33087009972618	0.0156260932199399\\
1.92697586132023	0.0156929341853527\\
};
\addlegendentry{$N =80$}

\end{axis}
\end{tikzpicture}%
								}
							}
						\end{figure}
						RIScatter backscatter rate is lower than symbiotic radio (due to energy detection) but higher than ambient backscatter (due to adaptive encoding).
						% Throughput depends on the spreading factor $N$.
						For backscatter link, a large spreading factor improves the bit error rate but reduces the gross data rate.
					\end{exampleblock}

					\begin{block}{Conclusion}
						\begin{itemize}\setlength\itemsep{20pt}
							\item Cognitive active and passive transmission can benefit each other
							\item RIScatter nodes recycle ambient signal for modulation and beamforming
							\item No interference cancellation is required at the co-located receiver
							\item The key is to render the input distribution as a joint function of the information source, channel state information, and priority of coexisting links
						\end{itemize}
					\end{block}
				\end{column}
			\end{columns}
		\end{column}

		\separatorcolumn
	\end{columns}

	% \begin{columns}[t]

	% 	\begin{column}{2\colwidth+\sepwidth}
	% 		\begin{block}{Overview \& Motivation}
	% 			\begin{itemize}
	% 				\item What is backscatter communication?
	% 				\item What is reconfigurable intelligent surface?
	% 			\end{itemize}
	% 		\end{block}
	% 	\end{column}

	% \end{columns}

	% \begin{columns}[t]
	% 	\separatorcolumn

	% 	\begin{column}{\colwidth}

	% 		\begin{thm}{A \texttt{tcolorbox} theorem block containing a list}{}

	% 			Nam vulputate nunc felis, non condimentum lacus porta ultrices. Nullam sed
	% 			sagittis metus. Etiam consectetur gravida urna quis suscipit.

	% 			\begin{itemize}
	% 				\item \textbf{Mauris tempor} risus nulla, sed ornare
	% 				\item \textbf{Libero tincidunt} a duis congue vitae
	% 				\item \textbf{Dui ac pretium} morbi justo neque, ullamcorper
	% 			\end{itemize}

	% 			Eget augue porta, bibendum venenatis tortor.

	% 		\end{thm}

	% 		\begin{alertblock}{A highlighted block}

	% 			This block catches your eye, so \textbf{important stuff} should probably go
	% 			here.

	% 			Curabitur eu libero vehicula, cursus est fringilla, luctus est. Morbi
	% 			consectetur mauris quam, at finibus elit auctor ac. Aliquam erat volutpat.
	% 			Aenean at nisl ut ex ullamcorper eleifend et eu augue. Aenean quis velit
	% 			tristique odio convallis ultrices a ac odio.

	% 			\begin{itemize}
	% 				\item \textbf{Fusce dapibus tellus} vel tellus semper finibus. In
	% 				consequat, nibh sed mattis luctus, augue diam fermentum lectus.
	% 				\item \textbf{In euismod erat metus} non ex. Vestibulum luctus augue in
	% 				mi condimentum, at sollicitudin lorem viverra.
	% 			\end{itemize}

	% 			\begin{enumerate}
	% 				\item \textbf{Morbi mauris purus}, egestas at vehicula et, convallis
	% 				accumsan orci. Orci varius natoque penatibus et magnis dis parturient
	% 				montes, nascetur ridiculus mus.
	% 				\item \textbf{Cras vehicula blandit urna ut maximus}. Aliquam blandit nec
	% 				massa ac sollicitudin. Curabitur cursus, metus nec imperdiet bibendum,
	% 				velit lectus faucibus dolor, quis gravida metus mauris gravida turpis.
	% 			\end{enumerate}

	% 			Aenean tincidunt risus eros, at gravida lorem sagittis vel. Vestibulum ante
	% 			ipsum primis in faucibus orci luctus et ultrices posuere cubilia Curae.

	% 		\end{alertblock}

	% 		\begin{defbox}{A block containing an enumerated list}{}

	% 			Vivamus congue volutpat elit non semper. Praesent molestie nec erat ac
	% 			interdum. In quis suscipit erat. \textbf{Phasellus mauris felis, molestie
	% 				ac pharetra quis}, tempus nec ante. Donec finibus ante vel purus mollis
	% 			fermentum. Sed felis mi, pharetra eget nibh a, feugiat eleifend dolor. Nam
	% 			mollis condimentum purus quis sodales. Nullam eu felis eu nulla eleifend
	% 			bibendum nec eu lorem. Vivamus felis velit, volutpat ut facilisis ac,
	% 			commodo in metus.

	% 			\begin{enumerate}
	% 				\item \textbf{Morbi mauris purus}, egestas at vehicula et, convallis
	% 				accumsan orci. Orci varius natoque penatibus et magnis dis parturient
	% 				montes, nascetur ridiculus mus.
	% 				\item \textbf{Cras vehicula blandit urna ut maximus}. Aliquam blandit nec
	% 				massa ac sollicitudin. Curabitur cursus, metus nec imperdiet bibendum,
	% 				velit lectus faucibus dolor, quis gravida metus mauris gravida turpis.
	% 				\item \textbf{Vestibulum et massa diam}. Phasellus fermentum augue non
	% 				nulla accumsan, non rhoncus lectus condimentum.
	% 			\end{enumerate}

	% 		\end{defbox}

	% 		\begin{block}{Fusce aliquam magna velit}

	% 			Et rutrum ex euismod vel. Pellentesque ultricies, velit in fermentum
	% 			vestibulum, lectus nisi pretium nibh, sit amet aliquam lectus augue vel
	% 			velit.

	% 			\begin{enumerate}
	% 				\item \textbf{Morbi mauris purus}, egestas at vehicula et, convallis
	% 				accumsan orci. Orci varius natoque penatibus et magnis dis parturient
	% 				montes, nascetur ridiculus mus.
	% 				\item \textbf{Cras vehicula blandit urna ut maximus}. Aliquam blandit nec
	% 				massa ac sollicitudin. Curabitur cursus, metus nec imperdiet bibendum,
	% 				velit lectus faucibus dolor, quis gravida metus mauris gravida turpis.
	% 			\end{enumerate}

	% 			\begin{figure}
	% 				\centering
	% 				\begin{tikzpicture}
	% 					\begin{axis}[
	% 							scale only axis,
	% 							no markers,
	% 							domain=0:2*pi,
	% 							samples=100,
	% 							axis lines=center,
	% 							axis line style={-},
	% 							ticks=none]
	% 						\addplot[red] {sin(deg(x))};
	% 						\addplot[blue] {cos(deg(x))};
	% 					\end{axis}
	% 				\end{tikzpicture}
	% 				\caption{Another figure caption.}
	% 			\end{figure}

	% 		\end{block}

	% 	\end{column}

	% 	\separatorcolumn

	% 	\begin{column}{\colwidth}

	% 		\begin{block}{Nam cursus consequat egestas}

	% 			\begin{itemize}
	% 				\item \textbf{Sed consequat} id ante vel efficitur. Praesent congue massa
	% 				sed est scelerisque, elementum mollis augue iaculis.
	% 				\begin{itemize}
	% 					\item In sed est finibus, vulputate
	% 					nunc gravida, pulvinar lorem. In maximus nunc dolor, sed auctor eros
	% 					porttitor quis.
	% 					\item Fusce ornare dignissim nisi. Nam sit amet risus vel lacus
	% 					tempor tincidunt eu a arcu.
	% 					\item Donec rhoncus vestibulum erat, quis aliquam leo
	% 					gravida egestas.
	% 				\end{itemize}
	% 				\item \textbf{Sed luctus, elit sit amet} dictum maximus, diam dolor
	% 				faucibus purus, sed lobortis justo erat id turpis.
	% 				\item \textbf{Pellentesque facilisis dolor in leo} bibendum congue.
	% 				Maecenas congue finibus justo, vitae eleifend urna facilisis at.
	% 			\end{itemize}

	% 		\end{block}

	% 		\begin{exampleblock}{An example block containing some math}{}

	% 			Nullam non est elit. In eu ornare justo. Maecenas porttitor sodales lacus,
	% 			ut cursus augue sodales ac.

	% 			$$
	% 				\int_{-\infty}^{\infty} e^{-x^2}\,dx = \sqrt{\pi}
	% 			$$

	% 			\begin{enumerate}
	% 				\item \textbf{Morbi mauris purus}, egestas at vehicula et, convallis
	% 				accumsan orci. Orci varius natoque penatibus et magnis dis parturient
	% 				montes, nascetur ridiculus mus.
	% 				\item \textbf{Cras vehicula blandit urna ut maximus}. Aliquam blandit nec
	% 				massa ac sollicitudin. Curabitur cursus, metus nec imperdiet bibendum,
	% 				velit lectus faucibus dolor, quis gravida metus mauris gravida turpis.
	% 			\end{enumerate}

	% 			\heading{A heading inside a block}

	% 			Praesent consectetur mi $x^2 + y^2$ metus, nec vestibulum justo viverra
	% 			nec. Proin eget nulla pretium, egestas magna aliquam, mollis neque. Vivamus
	% 			dictum $\mathbf{u}^\intercal\mathbf{v}$ sagittis odio, vel porta erat
	% 			congue sed. Maecenas ut dolor quis arcu auctor porttitor.

	% 			\heading{Another heading inside a block}

	% 			Sed augue erat, scelerisque a purus ultricies, placerat porttitor neque.
	% 			Donec $P(y \mid x)$ fermentum consectetur $\nabla_x P(y \mid x)$ sapien
	% 			sagittis egestas. Duis eget leo euismod nunc viverra imperdiet nec id
	% 			justo.

	% 		\end{exampleblock}

	% 		\begin{block}{Nullam vel erat at velit convallis laoreet}

	% 			Class aptent taciti sociosqu ad litora torquent per conubia nostra, per
	% 			inceptos himenaeos. Phasellus libero enim, gravida sed erat sit amet,
	% 			scelerisque congue diam. Fusce dapibus dui ut augue pulvinar iaculis.

	% 			\begin{table}
	% 				\centering
	% 				\begin{tabular}{l r r c}
	% 					\toprule
	% 					\textbf{First column} & \textbf{Second column} & \textbf{Third column} & \textbf{Fourth} \\
	% 					\midrule
	% 					Foo                   & 13.37                  & 384,394               & $\alpha$        \\
	% 					Bar                   & 2.17                   & 1,392                 & $\beta$         \\
	% 					Baz                   & 3.14                   & 83,742                & $\delta$        \\
	% 					Qux                   & 7.59                   & 974                   & $\gamma$        \\
	% 					\bottomrule
	% 				\end{tabular}
	% 				\caption{A table caption.}
	% 			\end{table}

	% 			Donec quis posuere ligula. Nunc feugiat elit a mi malesuada consequat. Sed
	% 			imperdiet augue ac nibh aliquet tristique. Aenean eu tortor vulputate,
	% 			eleifend lorem in, dictum urna. Proin auctor ante in augue tincidunt
	% 			tempor. Proin pellentesque vulputate odio, ac gravida nulla posuere
	% 			efficitur. Aenean at velit vel dolor blandit molestie. Mauris laoreet
	% 			commodo quam, non luctus nibh ullamcorper in. Class aptent taciti sociosqu
	% 			ad litora torquent per conubia nostra, per inceptos himenaeos.

	% 			Nulla varius finibus volutpat. Mauris molestie lorem tincidunt, iaculis
	% 			libero at, gravida ante. Phasellus at felis eu neque suscipit suscipit.
	% 			Integer ullamcorper, dui nec pretium ornare, urna dolor consequat libero,
	% 			in feugiat elit lorem euismod lacus. Pellentesque sit amet dolor mollis,
	% 			auctor urna non, tempus sem.

	% 		\end{block}


	% 		\begin{block}{References}
	% 			% \printbibliography[heading=none]
	% 		\end{block}
	% 	\end{column}

	% 	\separatorcolumn
	% \end{columns}
\end{frame}
\end{document}
